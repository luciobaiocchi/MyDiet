\documentclass[a4paper,12pt]{report}

\usepackage[italian]{babel}
\usepackage{hyperref}
\usepackage[utf8]{inputenc}
\usepackage{float}
\usepackage[table,xcdraw]{xcolor}
\usepackage{longtable}
\usepackage{graphicx}
\usepackage{todonotes}
\usepackage{fancyvrb}

\title{\textbf{CineRadar}\\Progetto di Basi di dati\\\textit{Università di Bologna}}
\author{Martin Tomassi; 0001077737\\Francesco Pazzaglia; 0001077423\\Luca Casadei; 0001069237}
\date{\today}

\begin{document}
\maketitle
\tableofcontents
\chapter{Analisi dei requisiti}
\section{Introduzione}
Il progetto consiste nella realizzazione di un applicativo per la condivisione e la recensione di elementi multimediali quali film e serie TV.
\section{Intervista}
È richiesto un sistema che consenta all'utenza di accedere al portale di condivisione di film e serie TV e poter recensirne uno o più di interesse, nel dettaglio, un film è un elemento multimediale unico, mentre una serie TV è composta da più stagioni, composte a loro volta da un certo numero di episodi. È necessario che l'utente si registri inserendo i propri dati che verranno salvaguardati, tra cui username, password, nome, cognome, data di nascita per controllare i limiti di età sui film. Un utente deve avere la possibilità di aggiungere nuovi film o serie nel sistema, ma non in maniera diretta, la richiesta deve prima essere approvata da un altro tipo di utente con privilegi di amministratore del sistema. L'amministratore si occuperà di aggiungere film alla piattaforma con frequenza settimanale. Sulle recensioni di altri utenti, un utente può esprimere un parere di utilità positivo o negativo che andrà a classificare la recensione in una sorta di "classifica delle recensioni più utili".
\\
Quando si va a memorizzare un film o una serie TV bisogna inserire il titolo, uno o più autori (che possono essere registi o attori), la descrizione, la durata e l'anno di rilascio. Gli utenti devono essere in grado di vedere tutti gli elementi multimediali presenti, in base a dei filtri sul genere, sempre a patto che i risultati ottenuti rispettino i limiti di età standard dei film/serie (\textit{over 18, over 16} etc\dots). Quando un utente vede un film della lista, deve poterlo segnare come \textit{"già visto"} per poi poterlo recensire se lo desidera, mentre una stagione di una serie TV è considerata "vista" se tutti i suoi episodi sono stati contrassegnati come visti. Dato che si vogliono anche incentivare le famiglie, è richiesta una funzionalità che renda possibile inserire un'età e visualizzare i film/serie la cui visione è consentita (per esempio per capire se un film/serie è adatto per il proprio figlio). Un utente registrato all'interno dell'applicativo può ricevere, in base alla quantità e qualità delle sue recensioni, dei coupon da utilizzare all'interno di un cinema che ha erogato tale sconto, dopo aver ottenuto la loro tessera da affiliato. Questi premi vengono assegnati agli utenti da parte dell'amministratore del sistema, mentre un registratore si occuperà di inserire le tessere di un cinema.\\ 
Il coupon può essere utilizzato per un elemento multimediale o genere specifico entro la sua data di scadenza. Sarà compito dei cinema stabilire quale impiegato avrà la facoltà di erogare le tessere e diventare così registratore sulla piattaforma. Per tenere traccia delle strutture che hanno emesso delle tessere vengono memorizzati i cinema attraverso un codice numerico, un indirizzo ed un nome. All'interno di ogni cinema è presente uno o più addetti registratori che sono in grado di registrare nuove tessere per il cinema al quale afferiscono.
Vi devono essere anche delle piccole classifiche per incentivare gli utenti ad effettuare recensioni, per esempio una classifica degli utenti con il maggior numero di film/serie visualizzati o recensiti. Vi è inoltre una statistica sui film/serie inseriti, per esempio sarà possibile ottenere il genere di film/serie con il maggior numero di visualizzazioni complessive. \\ 
L'amministratore effettuerà anche atti di moderazione sugli utenti registrati, ad esempio, se un utente effettua troppe recensioni al di sotto della soglia di utilità potrà essere rimosso o bloccato dall'effettuare l'accesso al sistema.
\section{Rielaborazione del testo}
\subsection{Obiettivi finali:}
È richiesto un sistema che consenta all'utenza di accedere al portale di condivisione di \textcolor{orange}{\underline{film}} e \textcolor{orange}{\underline{serie TV}} per poterne recensire uno o più di interesse.\\
Vi devono essere anche delle piccole \textcolor{orange}{\underline{classifiche}} per incentivare gli utenti ad effettuare \textcolor{orange}{\underline{recensioni}}, per esempio una classifica degli utenti con il maggior numero di film visualizzati o recensiti.\\
Vi è inoltre una statistica sui film inseriti, per esempio sarà possibile ottenere il \textcolor{orange}{\underline{genere}} di film con il maggior numero di visualizzazioni complessive. Inoltre è richiesta anche una statistica sugli \textcolor{orange}{\underline{attori}} e i \textcolor{orange}{\underline{registi}} che hanno prodotto i film inseriti, esempio, l'attore con più comparse all'interno dei film inseriti.
Un utente registrato sull'applicativo può ricevere, in base alla quantità e qualità delle sue recensioni, dei coupon da utilizzare all'interno di un cinema che ha erogato tale sconto, dopo aver ottenuto la loro tessera da affiliato. Il coupon può essere utilizzato per un film o genere specifico entro la sua data di scadenza.
\subsection{Funzionalità lato utente:}
È necessario che l'\textcolor{orange}{\underline{utente}} si registri inserendo i propri dati che verranno salvaguardati, tra cui username, password, nome, cognome, data di nascita per controllare i limiti di età sui film. Un utente deve avere la possibilità di aggiungere nuovi \textcolor{orange}{\underline{film}} attraverso una richiesta al sistema, essa deve prima essere approvata da un altro tipo di utente con privilegi di \textcolor{orange}{\underline{amministratore}} del sistema.\\
Quando un utente vede un film della lista, deve poterlo segnare come \textcolor{orange}{\underline{"Visto"}} per poterlo recensire successivamente se lo desidera. Dato che si vogliono anche incentivare le \textcolor{orange}{\underline{famiglie}}, è richiesta una funzionalità che renda possibile inserire un'età e visualizzare i film/serie la cui visione è consentita (per esempio per capire se un film è adatto per il proprio figlio).\\
Gli utenti devono essere in grado di vedere tutti i film/serie presenti in base a dei \textcolor{orange}{\underline{filtri}} sul genere, categoria, autori e recensioni, sempre a patto che i risultati ottenuti rispettino i \textcolor{orange}{\underline{limiti di età}} standard dei film/serie (\textit{over 18, over 16} etc\dots).
\subsection{Funzionalità lato amministrativo:}
L'\textcolor{orange}{\underline{amministratore}} si occuperà di aggiungere film/serie alla piattaforma con frequenza settimanale. Sulle recensioni di altri utenti, un utente può esprimere un parere di utilità che andrà a classificare la recensione in una sorta di "classifica delle recensioni più utili".\\
Quando si va a memorizzare un film/serie bisogna inserire il titolo, uno o più autori, la descrizione, la durata, l'anno di rilascio ed eventuale casa produttrice.
L'amministratore può attribuire dei \textcolor{orange}{\underline{premi}} in base alla classifica degli utenti migliori (in base alle recensioni più utili), effettuerà anche atti di moderazione sugli utenti registrati, ad esempio, se un utente effettua troppe recensioni al di sotto della soglia di utilità potrà essere rimosso o bloccato dall'effettuare l'accesso al sistema.
\subsection{Funzionalità lato registratore:}
All'interno di ogni cinema è presente uno o più addetti \textcolor{orange}{\underline{registratori}} che sono in grado di registrare nuove tessere per il cinema al quale afferiscono.

\subsection{Termini da chiarire}\label{ss:terminologia}
\begin{itemize}
	\item "Utente" $\longrightarrow$ Un utilizzatore dell'applicativo che si registra (o che accede) alla piattaforma ed ha come compito principale la visione, scrittura e valutazione di recensioni.
	\item "Visto" $\longrightarrow$ Un utente può spuntare una casella "visto" se ha visto effettivamente il film o un episodio di una serie.
	\item "Filtro" $\longrightarrow$ Un filtro viene applicato sulla ricerca che può fare un utente sulla lista di film/serie, questo può riguardare l'autore, il titolo, le recensioni dell'utenza etc\dots
	\item "Classifica" $\longrightarrow$ Una lista visualizzabile sull'applicativo in base a parametri scelti; il concetto non viene esplicitato nello schema concettuale ma sarà realizzato attraverso delle query.
	\item "Famiglia" $\longrightarrow$ Un utente genitore può effettuare una sorta di controllo parentale inserendo un'età in un'apposita casella e visualizzando tutti i film/serie che rispettano il limite di età inserito.
\end{itemize}
\section{Estrazione dei dati del testo rielaborato}
\subsection{Estrazione dati sugli utenti}
\textbf{Utente} $\longrightarrow$ Utente che si registra sull'applicativo\\Successivamente verranno elencati i dati da dover memorizzare.
\begin{itemize}
	\item Username
	\item Password
	\item Nome
	\item Cognome
	\item Data di nascita
	\item Targa premio (opzionale)
\end{itemize}
\textbf{Amministratore} $\longrightarrow$ Utente privilegiato che effettua operazioni di moderazione e inserimento dati. Dato che nell'intervista non emergono i dati da memorizzare per l'amministratore, si presuppone che vi sia un contatto per poterlo raggiungere, oltre che alle credenziali.
\begin{itemize}
	\item Username
	\item Password
	\item Nome
	\item Cognome
	\item Telefono
\end{itemize}
\subsection{Estrazione dati sui film/serie e produttori}
\textbf{Film / Serie TV} $\longrightarrow$ Elementi multimediali su cui gli utenti registrati possono apporre la propria visualizzazione ed effettuare recensioni in seguito.
\begin{itemize}
	\item Data di rilascio
	\item Titolo
	\item Descrizione
	\item Genere
	\item Limite di età
\end{itemize}
\textbf{Attori} $\longrightarrow$ Questi vengono memorizzati per stilare la classifica visualizzabile, ed esplicitare delle preferenze. Fanno parte del cast di un film.
\begin{itemize}
	\item Nome
	\item Cognome
	\item Nome d'arte
	\item Data di nascita
\end{itemize}
\textbf{Registi} $\longrightarrow$ Sono coloro che dirigono il cast.
\begin{itemize}
	\item Nome
	\item Cognome
	\item Data di nascita
	\item Debutto carriera
\end{itemize}
\subsection{Elenco delle azioni}
\subsubsection{Utente}
\begin{itemize}
	\item Registrarsi sulla piattaforma.
	\item Accedere alla piattaforma.
	\item Scegliere le proprie categorie di preferenza.
	\item Visualizzare i film/serie in base ai filtri scelti e ai limiti di età.
	\item Data un età inferiore alla propria, ottenere la lista dei film che sono visualizzabili con quell'età.
	\item Richiedere all'amministratore l'aggiunta di un film / serie non presente in elenco.
	\item Contrassegnare come visualizzato un film / episodio di una serie.
	\item Recensire i film contrassegnati come visualizzati.
	\item Dare una valutazione di utilità alle recensioni degli altri utenti.
	\item Ricercare dei film in base al genere o all'autore.
	\item Visualizzare le classifiche degli autori e dei generi.
\end{itemize}
\subsubsection{Amministratore}
\begin{itemize}
	\item Ottenere le statistiche degli utenti registrati alla piattaforma per poter effettuare attività di moderazione, tra cui:
	      \begin{itemize}
		      \item (S)Bloccare o eliminare l'accesso ad un utente alla piattaforma se ha effettuato troppe recensioni che stanno sotto alla soglia di utilità.
		      \item Premiare l'utente in cima alla classifica delle recensioni più utili (il premio è una targhetta che comparirà accanto al nome).
	      \end{itemize}
	\item Aggiungere film / serie alla lista, compresi quelli che sono stati richiesti dagli utenti.
	\item Inserire nuovi autori o registi da associare a dei film.
\end{itemize}
\subsubsection{Registratore}
\begin{itemize}
	\item Aggiungere nuove tessere per il cinema al quale afferisce.
\end{itemize}
\chapter{Progettazione concettuale}
\section{Strategia bottom-up}
\subsection{Utenza}
Rappresentazione di una sottoparte dello schema ER che riguarda la gestione dell'utenza, in particolare, è stata pensata una generalizzazione dell'entità \textbf{utilizzatore} con due sotto-entità \textbf{amministratore} e \textbf{utente} di cui dobbiamo memorizzare elementi diversi. La generalizzazione è di tipo \textit{totale} e \textit{esclusiva}, questo perché l'amministratore ha un tipo di account diverso da quello di semplice utente, se l'amministratore vuole effettuare le operazioni di un utente normale deve registrarsi con un altro account.
\begin{figure}[H]
	\centering
	\includegraphics[width=300pt]{ER/utenza.png}
	\caption{Schema ER dell'utenza.}
\end{figure}
\subsection{Multimedia}
Qui viene descritto il concetto di \textbf{multimedia}(riferimento nella sezione \ref{ss:terminologia}), che si estende tramite generalizzazione alla rappresentazione di due sotto entità: \textbf{film} e \textbf{serie tv}. Un multimedia può essere associato a diversi \textbf{generi}, \textit{(quindi possono esistere, ad esempio, film/serie che sono sia horror che commedie)}, e l'inserimento dei generi è indipendente dall'esistenza dei multimedia, quindi si è optato per la cardinalità \textbf{0-N}. La principale differenza tra un film e una serie TV sta nel fatto che una serie può essere composta da più stagioni, ognuna con diversi episodi, mentre un film è un'unica narrazione senza interruzioni.
\begin{figure}[H]
	\centering
	\includegraphics[width=300pt]{ER/multimedia.png}
	\caption{Schema ER dei multimedia.}
\end{figure}
\subsection{Cast}
I membri del cast da tracciare possono essere \textbf{registi} o \textbf{attori}; degli attori ci interessa in particolare un nome d'arte, mentre dei registi la data di debutto della carriera per effettuare delle statistiche. La generalizzazione in questo caso è \textit{totale} e \textit{sovrapposta}, questo significa che nel dominio del nostro problema non si vogliono tracciare altri membri oltre che a questi due, ed è sovrapposta perché possono esserci casi in cui un regista è anche un attore.
\begin{figure}[H]
	\centering
	\includegraphics[width=300pt]{ER/cast.png}
	\caption{Schema ER dei membri del cast.}
\end{figure}
\subsection{Recensione}
Una recensione è suddivisa in più sezioni(ad esempio: trama, colonna sonora, etc\dots) e per ciascuna di esse viene assegnato un voto. Una sezione, invece, può esistere anche senza il concetto di recensione.
Il concetto di recensione si suddivide in due sottoconcetti, mediante una generalizzazione: recfilm (recensione di film) e recserie (recensione di serie TV). Entrambi sono identificati dall'utente che ha redatto la recensione, tuttavia recfilm è identificato anche dal film trattato, mentre recserie è identificata anche dalla serie televisiva.
\begin{figure}[H]
	\centering
	\includegraphics[width=300pt]{ER/recensione.png}
	\caption{Schema ER della recensione.}
\end{figure}
\subsection{Promo}
Una promozione è un'offerta limitata da una scadenza specifica, viene definita come un'istanza di un modello promozionale. Tale \textit{template} è caratterizzato da una percentuale di sconto, che gli utenti possono usufruire. La promo \textit{template} è suddivisa in due sotto entità in maniera totale ed esclusiva:
\begin{itemize}
	\item Singolo: valido per uno e un solo film.
	\item Multiplo: valido per uno o più generi di film.
\end{itemize}
\begin{figure}[H]
	\centering
	\includegraphics[width=270pt]{ER/promo.png}
	\caption{Schema ER della promo.}
\end{figure}
\subsection{Relazione tra utenza e recensioni}
Un utente può scrivere più di una recensione su vari film o serie tv, esplicitando il titolo, la descrizione della recensione e la sua valutazione. Quando viene scritta una recensione vengono scelti dei voti, ciascuno per ogni sezione selezionata dall'utente.
\begin{figure}[H]
	\centering
	\includegraphics[width=300pt]{ER/utenzarecensione.png}
	\caption{Schema ER della relazione tra utenza e recensioni.}
\end{figure}
\subsection{Relazione tra utenza e cinema}
Un \textbf{cinema} afferisce a più registratori, mentre ciascuno di essi è a disposizione solo su di un cinema in particolare. Il \textbf{registratore} è considerato un \textbf{utilizzatore}, quindi è compreso nella generalizzazione.
\begin{figure}[H]
	\centering
	\includegraphics[width=300pt]{ER/utenzacinema.png}
	\caption{Schema ER della relazione tra utenza e cinema.}
\end{figure}
\subsection{Relazione tra utenza e tessera}
Gli utenti hanno la possibilità di iscriversi a una o più \textbf{tessere}. Una tessera è inserita da un \textbf{registratore}. Ciascuna tessera è caratterizzata da un numero univoco, che la identifica insieme all'utente tesserato, e da una data di rinnovo.
\begin{figure}[H]
	\centering
	\includegraphics[width=300pt]{ER/utenzatessera.png}
	\caption{Schema ER della relazione tra utenza e tessera.}
\end{figure}
\subsection{Relazione tra multimedia e membri del cast}
Un \textbf{cast} può essere diretto da uno e un solo \textbf{regista} mentre possono prendere parte all'interno dello stesso uno o più \textbf{attori}. Queste due entità sono raggruppate tramite la generalizzazione "membro cast" che descrive un qualsiasi membro del cast cinematografico che ha presto parte nel film stesso.
Il cast è specifico per ogni singolo film, mentre può variare da stagione a stagione all'interno di una serie TV.
\begin{figure}[H]
	\centering
	\includegraphics[width=350pt]{ER/multimediacast.png}
	\caption{Schema ER della relazione tra multimedia e membri del cast}
\end{figure}
\subsection{Relazione tra multimedia e promo}
Un multimedia può essere oggetto di offerte promozionali sia direttamente che tramite il suo genere. Il modello di promozione viene suddiviso in singolo e multiplo attraverso una generalizzazione, distinguendo il primo che è specifico del multimedia stesso, mentre il secondo si riferisce al genere.
\begin{figure}[H]
	\centering
	\includegraphics[width=300pt]{ER/multimediapromo.png}
	\caption{Schema ER della relazione tra multimedia e promo.}
\end{figure}
\subsection{Relazione tra multimedia, cast e promo}
Di un cast vengono memorizzati i membri che, mediante una generalizzazione, sono suddivisi in regista e attori. Il cast è associato a un film particolare o a una stagione di una serie TV. Queste due entità sono aggregate dal concetto di multimedia, il quale, come precedentemente descritto, è collegato al sistema di offerte promozionali.
\begin{figure}[H]
	\centering
	\includegraphics[width=300pt]{ER/multimediacastpromo.png}
	\caption{Schema ER della relazione tra multimedia, cast e promo.}
\end{figure}
\subsection{Relazione tra utenza, cinema, tessera e recensione}
Una tessera, inserita da un registratore per il cinema per cui lavora, può essere utilizzata da un utente che è stato premiato perché risultante particolarmente attivo sulla piattaforma. Questi utenti possono usufruire di vantaggi quali sconti e offerte per mezzo della tessera stessa. Nello specifico una tessera è valida solo per un cinema, quello di afferenza del registratore.
Inoltre includiamo il collegamento che vi è tra l'utenza e la recensione, che è stata precedentemente descritta.
\begin{figure}[H]
	\centering
	\includegraphics[width=450pt]{ER/utenzacinematesserarecensione.png}
	\caption{Schema ER della relazione tra utenza, cinema, tessera e recensione.}
\end{figure}
\subsection{Schema completo}
Quello che segue rappresenta lo schema derivato dalla composizione di tutte le sotto-componenti descritte precedentemente.
\begin{figure}[H]
	\centering
	\includegraphics[width=450pt]{ER/ercompletosx.png}
	\caption{Schema ER completo \textbf{1/2}.}
\end{figure}
\begin{figure}[H]
	\centering
	\includegraphics[width=450pt]{ER/ercompletodx.png}
	\caption{Schema ER completo \textbf{2/2}.}
\end{figure}
\chapter{Progettazione Logica}
\section{Stima del volume dei dati} \label{s:volumes}
\begin{table}[H]
	\centering
	\begin{tabular}{|lll|}
		\hline
		\rowcolor[HTML]{FFCE93}
		\multicolumn{3}{|l|}{\cellcolor[HTML]{FFCE93}Film} \\ \hline
		\rowcolor[HTML]{CBCEFB}
		Concetto            & Costrutto & Volume           \\ \hline
		MULTIMEDIA          & E         & 12000            \\ \hline
		FILM                & E         & 9500             \\ \hline
		SERIE               & E         & 2500             \\ \hline
		GENERE              & E         & 13               \\ \hline
		categorizzazione    & R         & 50000            \\ \hline
		visualizzazionefilm & R         & 350000           \\ \hline
		preferenza          & R         & 29880            \\ \hline
	\end{tabular}
\end{table}
\begin{table}[H]
	\centering
	\begin{tabular}{|lll|}
		\hline
		\rowcolor[HTML]{FFCE93}
		\multicolumn{3}{|l|}{\cellcolor[HTML]{FFCE93}Cast} \\ \hline
		\rowcolor[HTML]{CBCEFB}
		Concetto    & Costrutto & Volume                   \\ \hline
		ATTORE      & E         & 5200                     \\ \hline
		REGISTA     & E         & 600                      \\ \hline
		CAST        & E         & 12000                    \\ \hline
		attorecast  & R         & 12000                    \\ \hline
		registacast & R         & 12000                    \\ \hline
	\end{tabular}
\end{table}
\begin{table}[H]
	\centering
	\begin{tabular}{|lll|}
		\hline
		\rowcolor[HTML]{FFCE93}
		\multicolumn{3}{|l|}{\cellcolor[HTML]{FFCE93}Serie TV} \\ \hline
		\rowcolor[HTML]{CBCEFB}
		Concetto          & Costrutto & Volume                 \\ \hline
		STAGIONE          & E         & 10000                  \\ \hline
		EPISODIO          & E         & 400000                 \\ \hline
		compstagione      & R         & 400000                 \\ \hline
		compserie         & R         & 10000                  \\ \hline
		visualizzazioneep & R         & 1400000                \\ \hline
	\end{tabular}
\end{table}
\begin{table}[H]
	\centering
	\begin{tabular}{|lll|}
		\hline
		\rowcolor[HTML]{FFCE93}
		\multicolumn{3}{|l|}{\cellcolor[HTML]{FFCE93}Utilizzatori} \\ \hline
		\rowcolor[HTML]{CBCEFB}
		Concetto       & Costrutto & Volume                        \\ \hline
		UTILIZZATORE   & E         & 10000                         \\ \hline
		AMMINISTRATORE & E         & 10                            \\ \hline
		REGISTRATORE   & E         & 30                            \\ \hline
		UTENTE         & E         & 9960                          \\ \hline
	\end{tabular}
\end{table}
\begin{table}[H]
	\centering
	\begin{tabular}{|lll|}
		\hline
		\rowcolor[HTML]{FFCE93}
		\multicolumn{3}{|l|}{\cellcolor[HTML]{FFCE93}Cinema e registratori di tessere} \\ \hline
		\rowcolor[HTML]{CBCEFB}
		Concetto     & Costrutto & Volume                                              \\ \hline
		CINEMA       & E         & 25                                                  \\ \hline
		TESSERA      & E         & 3500                                                \\ \hline
		appartenenza & R         & 3500                                                \\ \hline
		tesseramento & R         & 3500                                                \\ \hline
		afferenza    & R         & 30                                                  \\ \hline
	\end{tabular}
\end{table}
\begin{table}[H]
	\centering
	\begin{tabular}{|lll|}
		\hline
		\rowcolor[HTML]{FFCE93}
		\multicolumn{3}{|l|}{\cellcolor[HTML]{FFCE93}Coupon e premi su tessera} \\ \hline
		\rowcolor[HTML]{CBCEFB}
		Concetto       & Costrutto & Volume                                     \\ \hline
		PROMO          & E         & 200                                        \\ \hline
		TEMPLATEPROMO  & E         & 500                                        \\ \hline
		MULTIPLO       & E         & 25                                         \\ \hline
		SINGOLO        & E         & 475                                        \\ \hline
		premitessera   & R         & 1500                                       \\ \hline
		validità       & R         & 200                                        \\ \hline
		couponsugenere & R         & 50                                         \\ \hline
		couponsufilm   & R         & 475                                        \\ \hline
	\end{tabular}
\end{table}
\begin{table}[H]
	\centering
	\begin{tabular}{|lll|}
		\hline
		\rowcolor[HTML]{FFCE93}
		\multicolumn{3}{|l|}{\cellcolor[HTML]{FFCE93}Recensioni} \\ \hline
		\rowcolor[HTML]{CBCEFB}
		Concetto                 & Costrutto & Volume            \\ \hline
		RECENSIONE               & E         & 1200000           \\ \hline
		SEZIONE                  & E         & 5                 \\ \hline
		recensionefilm           & R         & 950000            \\ \hline
		recensioneserie          & R         & 250000            \\ \hline
		valutazione              & R         & 115000            \\ \hline
		valutazionesezione       & R         & 1400000           \\ \hline
		scritturarecensionefilm  & R         & 950000            \\ \hline
		scritturarecensioneserie & R         & 250000            \\ \hline
	\end{tabular}
\end{table}
\section{Operazioni e loro frequenza}
\subsection{Operazioni dell'utenza}
\begin{longtable}[H]{|c|c|>{\columncolor[HTML]{FFFFC7}}c |c|}
	\hline
	\cellcolor[HTML]{ECF4FF}Numero                                                                                                                                                                               &
	\cellcolor[HTML]{ECF4FF}Operazione                                                                                                                                                                           &
	\cellcolor[HTML]{ECF4FF}Frequenza / gg                                                                                                                                                                       &
	\cellcolor[HTML]{ECF4FF}Dettagli                                                                                                                                                                                                                                                                         \\ \hline
	\endfirsthead
	%
	\endhead
	%
	1                                                                                                                                                                                                            &
	\begin{tabular}[c]{@{}c@{}}Registrazione di \\ un nuovo utente.\end{tabular}                                                                                                                                 &
	2                                                                                                                                                                                                            &
	\\ \hline
	2.1                                                                                                                                                                                                          &
	\begin{tabular}[c]{@{}c@{}}Accesso di un \\ utente.\end{tabular}                                                                                                                                             &
	100                                                                                                                                                                                                          &
	\\ \hline
	2.2                                                                                                                                                                                                          &
	\begin{tabular}[c]{@{}c@{}}Accesso di un \\ amministratore.\end{tabular}                                                                                                                                     &
	2                                                                                                                                                                                                            &
	\\ \hline
	2.3                                                                                                                                                                                                          &
	\begin{tabular}[c]{@{}c@{}}Accesso di un \\ registratore.\end{tabular}                                                                                                                                       &
	5                                                                                                                                                                                                            &
	\\ \hline
	3.1                                                                                                                                                                                                          &
	Scelta delle preferenze.                                                                                                                                                                                     &
	2                                                                                                                                                                                                            &
	Solo in fase di registrazione                                                                                                                                                                                                                                                                            \\ \hline
	3.2                                                                                                                                                                                                          &
	\begin{tabular}[c]{@{}c@{}}Aggiornamento \\ delle preferenze.\end{tabular}                                                                                                                                   &
	15                                                                                                                                                                                                           &
	\begin{tabular}[c]{@{}c@{}}Tra tutti gli utenti \\ che accedano in un \\ giorno, è plausibile \\ che pochi aggiornino \\ le proprie preferenze.\end{tabular}                                                                                                                                             \\ \hline
	4.1                                                                                                                                                                                                          &
	\begin{tabular}[c]{@{}c@{}}Visualizzare tutto \\ l'elenco dei film \\ in base all'età \\ di chi lo richiede.\end{tabular}                                                                                    &
	100                                                                                                                                                                                                          &
	\begin{tabular}[c]{@{}c@{}}Viene effettuato all'accesso.\end{tabular}                                                                                                                                                                                                                                    \\ \hline
	4.2                                                                                                                                                                                                          &
	\begin{tabular}[c]{@{}c@{}}Visualizzare l'elenco \\ dei film in base\\ ad un'età scelta.\end{tabular}                                                                                                        &
	17                                                                                                                                                                                                           &
	\begin{tabular}[c]{@{}c@{}}Meno frequenza \\ rispetto all'operazione\\ precedente, \\ i genitori registrati che\\ accedono ed usano \\ questa funzione sono\\ meno.\end{tabular}                                                                                                                         \\ \hline
	4.3                                                                                                                                                                                                          &
	\begin{tabular}[c]{@{}c@{}}Visualizzare l'elenco \\ delle serie TV \\ con la relativa \\ durata complessiva \\ e il numero  di stagioni \\ ed episodi complessivo, \\ in base all'età \\ di chi lo richiede.\end{tabular} &
	100                                                                                                                                                                                                          &
	\begin{tabular}[c]{@{}c@{}}Viene effettuato all'accesso.\end{tabular}                                                                                                                                                                                                                                    \\ \hline
	4.4                                                                                                                                                                                                          &
	\begin{tabular}[c]{@{}c@{}}Visualizzare le \\ informazioni riguardanti \\ una serie TV, comprese \\ tutte le stagioni \\ e tutti gli episodi.\end{tabular}                                                   &
	50                                                                                                                                                                                                           &
	\\ \hline
	5.1                                                                                                                                                                                                          &
	\begin{tabular}[c]{@{}c@{}}Contrassegnare \\ come "visualizzato" \\ un film.\end{tabular}                                                                                                                    &
	350                                                                                                                                                                                                          &
	\begin{tabular}[c]{@{}c@{}}Considerando che in \\ genere chi accede\\ all'applicativo e vede \\ la lista dei film,\\ scrive una recensione \\ su almeno un film,\\ implica che l'abbia \\ precedentemente\\ visualizzato. \\ Potrebbe essere \\ che venga contrassegnato \\ più di un film.\end{tabular} \\ \hline
	5.2                                                                                                                                                                                                          &
	\begin{tabular}[c]{@{}c@{}}Contrassegnare \\ come "visualizzato" \\ un episodio \\ di una serie.\end{tabular}                                                                                                &
	500                                                                                                                                                                                                          &                                                                                           \\ \hline
	6                                                                                                                                                                                                            &
	Recensire un film.                                                                                                                                                                                           &
	20                                                                                                                                                                                                           &
	\begin{tabular}[c]{@{}c@{}}Dei film contrassegnati \\ in un giorno solo\\ alcuni verranno recensiti.\end{tabular}                                                                                                                                                                                        \\ \hline
	7.1                                                                                                                                                                                                          &
	\begin{tabular}[c]{@{}c@{}}Visualizzare le \\ recensioni di un film.\end{tabular}                                                                                                                            &
	250                                                                                                                                                                                                           &
	\\ \hline
	7.2                                                                                                                                                                                                          &
	\begin{tabular}[c]{@{}c@{}}Dare una valutazione \\ di utilità ad una\\ recensione di un altro \\ utente su un film.\end{tabular}                                                                             &
	40                                                                                                                                                                                                           &
	\\ \hline
	7.3                                                                                                                                                                                                          &
	\begin{tabular}[c]{@{}c@{}}Dare una valutazione \\ di utilità ad una\\ recensione di un altro \\ utente su una serie.\end{tabular}                                                                           &
	30                                                                                                                                                                                                           &
	\\ \hline
	7.4                                                                                                                                                                                                          &
	\begin{tabular}[c]{@{}c@{}}Visualizzare le \\ recensioni di una \\ singola serie.\end{tabular}                                                                                                               &
	150                                                                                                                                                                                                           &
	\\ \hline
	7.5                                                                                                                                                                                                          &
	\begin{tabular}[c]{@{}c@{}}Aggiornare il voto \\ di una sezione \\ di una recensione.\end{tabular}                                                                                                               &
	50                                                                                                                                                                                                           &
	\\ \hline
	8.1                                                                                                                                                                                                          &
	\begin{tabular}[c]{@{}c@{}}Ottenere una \\ classifica dei generi\\ più visualizzati.\end{tabular}                                                                                                            &
	5                                                                                                                                                                                                            &
	\begin{tabular}[c]{@{}c@{}}Si considera nella \\ classifica solo la serie \\ completa come visualizzata, \\ se si visualizza \\ un singolo episodio \\ senza completare la \\ serie di appartenenza,\\ questo non verrà\\ considerato nel conteggio\\ per le visualizzazioni\\ dei generi.\end{tabular}
	\\ \hline
\end{longtable}
\subsection{Operazioni di amministrazione}
\begin{longtable}[H]{|c|c|>{\columncolor[HTML]{FFFFC7}}c |c|}
	\hline
	\cellcolor[HTML]{ECF4FF}Numero                                                                                                                                                   &
	\cellcolor[HTML]{ECF4FF}Operazione                                                                                                                                               &
	\cellcolor[HTML]{ECF4FF}Frequenza                                                                                                                                                &
	\cellcolor[HTML]{ECF4FF}Dettagli                                                                                                                                                                 \\ \hline
	\endfirsthead
	%
	\endhead
	%
	9.1                                                                                                                                                                              &
	\begin{tabular}[c]{@{}c@{}}Reperimento della\\ classifica degli \\ utenti con la media\\ delle valutazioni di\\ utilità sulle proprie\\ recensioni peggiore.\end{tabular}        &
	1 / mese                                                                                                                                                                         &
	\\ \hline
	9.2                                                                                                                                                                              &
	\begin{tabular}[c]{@{}c@{}}Come 9.1 ma è la\\ media delle recensioni\\ migliori.\end{tabular}                                                                                    &
	1 / settimana                                                                                                                                                                    &
	\begin{tabular}[c]{@{}c@{}}La premiazione degli\\ utenti con dei coupon\\ è settimanale.\end{tabular}                                                                                            \\ \hline
	10                                                                                                                                                                               &
	\begin{tabular}[c]{@{}c@{}}Assegnamento in blocco \\ di una promozione\\ ai primi 5 utenti tesserati \\ che si trovano in cima alla\\ classifica stilata (vedi 9.2)\end{tabular} &
	5 / settimana                                                                                                                                                                    &
	\\ \hline
	11.1                                                                                                                                                                             &
	\begin{tabular}[c]{@{}c@{}}Aggiunta di un nuovo \\ film alla piattaforma.\end{tabular}                                                                                           &
	4 / giorno                                                                                                                                                                       &
	\begin{tabular}[c]{@{}c@{}}Escono circa 2000 film \\ all'anno in tutto il mondo, \\ supponendo che si \\ aggiungano tutti i film\\ appena escono, vengono\\ circa 4 film al giorno.\end{tabular} \\ \hline
	11.2                                                                                                                                                                             &
	\begin{tabular}[c]{@{}c@{}}Aggiunta di persone che\\ hanno realizzato un film\\ alla piattaforma\end{tabular}                                                                    &
	664 / mese                                                                                                                                                                       &
	\begin{tabular}[c]{@{}c@{}}Considerando una stima\\ di 4 membri rilevanti del\\ cast di un film che si\\ vogliono tracciare, regista\\ compreso.\end{tabular}                                    \\ \hline
	11.3                                                                                                                                                                                                          &
	\begin{tabular}[c]{@{}c@{}}Aggiunta di una \\ nuova stagione per una \\ specifica serie TV.\end{tabular}                                                 &
	5 /
	settimana																				&
	
	\\ \hline
	11.4                                                                                                                                                                                                          &
	\begin{tabular}[c]{@{}c@{}}Aggiunta di un \\ nuovo episodio per una \\ specifica stagione di \\ una serie TV.\end{tabular}                                                   &
	10 /
	settimana																						&
	
	\\ \hline
\end{longtable}
\subsection{Operazioni del registratore}
\begin{longtable}[H]{|c|c|>{\columncolor[HTML]{FFFFC7}}c |c|}
	\hline
	\cellcolor[HTML]{ECF4FF}Numero                                                   &
	\cellcolor[HTML]{ECF4FF}Operazione                                               &
	\cellcolor[HTML]{ECF4FF}Frequenza                                                &
	\cellcolor[HTML]{ECF4FF}Dettagli                                                   \\ \hline
	\endfirsthead
	%
	\endhead
	%
	12                                                                               &
	\begin{tabular}[c]{@{}c@{}}Registrazione di \\ una nuova \\ tessera\end{tabular} &
	20 / giorno                                                                      &
	\\ \hline
\end{longtable}
\section{Raffinamento dello schema}
\subsection{Eliminazione delle gerarchie}
\subsubsection{	Gerarchia "Multimedia"}
Questa gerarchia è stata risolta adottando il metodo del collasso verso il basso, questo perché l'accesso alle entità avviene separatamente (i film vengono reperiti in maniera separata dalle serie e viceversa). Questo metodo è inoltre applicabile perché la copertura della gerarchia è totale ed esclusiva.
\begin{figure}[H]
	\centering
	\includegraphics{ER/ristrutturazione/ristmultimedia.png}
	\caption{Ristrutturazione della gerarchia "Multimedia"}
\end{figure}
\subsubsection{Gerarchia "Recensione"}
Analogamente alla precedente, è stato usato il collasso verso il basso.
\begin{figure}[H]
	\centering
	\includegraphics{ER/ristrutturazione/ristrecensione.png}
	\caption{Ristrutturazione della gerarchia "Recensione"}
\end{figure}
\subsubsection{Gerarchia "Membro Cast"}
In questo caso abbiamo una copertura totale ma sovrapposta, e dato che in genere quando si consulta il cast di un film, che è una consultazione più frequente rispetto al reperimento di un singolo attore o regista, si accedono entrambi i tipi di membri del cast contemporaneamente, per questo si è optato per il collasso verso l'alto con selettori di tipo (copertura sovrapposta).
\begin{figure}[H]
	\centering
	\includegraphics{ER/ristrutturazione/ristmembrocast.png}
	\caption{Ristrutturazione della gerarchia "Membro cast"}
\end{figure}
\subsubsection{Gerarchia "Template Promo"}
Questa gerarchia è stata rielaborata mantenendo le entità e introducendo delle relazione al posto della gerarchia, questo perché vi è copertura totale ed esclusiva e ci sono relazioni distinte sia con il padre che con le entità figlie.
\begin{figure}[H]
	\centering
	\includegraphics[width=400pt]{ER/ristrutturazione/ristpromo.png}
	\caption{Ristrutturazione della gerarchia "Template Promo"}
\end{figure}
\subsubsection{Gerarchia "Utilizzatore"}
Analogamente alla precedente, è stato usato il mantenimento delle gerarchie (Se si fosse utilizzato il collasso verso il basso i nomi utente non sarebbero stati univoci tra i diversi tipi di utente).
\begin{figure}[H]
	\centering
	\includegraphics{ER/ristrutturazione/ristutenza.png}
	\caption{Ristrutturazione della gerarchia "Utilizzatore"}
\end{figure}
\subsubsection{Schema completo dopo l'eliminazione delle gerarchie}
\begin{figure}[H]
	\centering
	\includegraphics[width=450pt]{ER/ristrutturazione/ristcomp1.png}
	\caption{Schema ristrutturato completo \textbf{1/2}}
\end{figure}
\begin{figure}[H]
	\centering
	\includegraphics[width=450pt]{ER/ristrutturazione/ristcomp2.png}
	\caption{Schema ristrutturato completo \textbf{2/2}}
\end{figure}

\section{Schemi di navigazione e tabelle degli accessi - Operazioni Utente}
Questi schemi gestiscono il caso senza alcuna ridondanza introdotta, le ridondanze verranno discusse nella sezione successiva con riferimenti a questa per le tabelle degli accessi senza ridondanza.
\subsubsection{Registrazione di un nuovo utente (Op. 1)}
\begin{figure}[H]
	\centering
	\includegraphics[width=450pt]{ER/navigazione/registrazioneutente.png}
	\caption{Schema di navigazione - registrazione nuovo utente}
\end{figure}
\begin{table}[H]
	\centering
	\begin{tabular}{|llll|}
		\hline
		\rowcolor[HTML]{CBCEFB}
		Concetto & Costrutto & Accesso & Tipo                             \\ \hline
		ACCOUNT  & E         & 1       & S                                \\ \hline
		UTENTE   & E         & 1       & S                                \\ \hline
		tipo.usr & R         & 1       & S                                \\ \hline
		\rowcolor[HTML]{CBCEFB}
		\multicolumn{4}{|l|}{\cellcolor[HTML]{FFCE93}\textbf{Totale}: 3S} \\ \hline
	\end{tabular}
\end{table}

\subsubsection{Accesso alla piattaforma di un utente (Op. 2.1)}
\begin{figure}[H]
	\centering
	\includegraphics[width=200pt]{ER/navigazione/accessoutente.png}
	\caption{Schema di navigazione - Accesso utente}
\end{figure}
\begin{table}[H]
	\centering
	\begin{tabular}{|llll|}
		\hline
		\rowcolor[HTML]{CBCEFB}
		Concetto & Costrutto & Accesso & Tipo                             \\ \hline
		ACCOUNT  & E         & 1       & L                                \\ \hline
		UTENTE   & E         & 1       & L                                \\ \hline
		tipo.usr & R         & 1       & L                                \\ \hline
		\rowcolor[HTML]{CBCEFB}
		\multicolumn{4}{|l|}{\cellcolor[HTML]{FFCE93}\textbf{Totale}: 3L} \\ \hline
	\end{tabular}
\end{table}

\subsection{Accesso alla piattaforma di un amministratore (Op. 2.2)}
\begin{figure}[H]
	\centering
	\includegraphics[width=250pt]{ER/navigazione/accessoamm.png}
	\caption{Schema di navigazione - Accesso amministratore}
\end{figure}
\begin{table}[H]
	\centering
	\begin{tabular}{|llll|}
		\hline
		\rowcolor[HTML]{CBCEFB}
		Concetto       & Costrutto & Accesso & Tipo                       \\ \hline
		ACCOUNT        & E         & 1       & L                          \\ \hline
		AMMINISTRATORE & E         & 1       & L                          \\ \hline
		tipo.amm       & R         & 1       & L                          \\ \hline
		\rowcolor[HTML]{CBCEFB}
		\multicolumn{4}{|l|}{\cellcolor[HTML]{FFCE93}\textbf{Totale}: 3L} \\ \hline
	\end{tabular}
\end{table}

\subsection{Accesso alla piattaforma di un registratore (Op. 2.3)}
\begin{figure}[H]
	\centering
	\includegraphics{ER/navigazione/accessoreg.png}
	\caption{Schema di navigazione - Accesso registratore}
\end{figure}
\begin{table}[H]
	\centering
	\begin{tabular}{|llll|}
		\hline
		\rowcolor[HTML]{CBCEFB}
		Concetto     & Costrutto & Accesso & Tipo                         \\ \hline
		ACCOUNT      & E         & 1       & L                            \\ \hline
		REGISTRATORE & E         & 1       & L                            \\ \hline
		tipo.reg     & R         & 1       & L                            \\ \hline
		\rowcolor[HTML]{CBCEFB}
		\multicolumn{4}{|l|}{\cellcolor[HTML]{FFCE93}\textbf{Totale}: 3L} \\ \hline
	\end{tabular}
\end{table}

\subsection{Visualizzare tutto l'elenco dei film in base all'età di chi lo richiede (Op. 4.1)}
\begin{figure}[H]
	\centering
	\includegraphics[width=300pt]{ER/navigazione/visualizzarefilm.png}
	\caption{Schema di navigazione - visualizzare elenco film}
\end{figure}
\begin{table}[H]
	\centering
	\begin{tabular}{|llll|}
		\hline
		\rowcolor[HTML]{CBCEFB}
		Concetto & Costrutto & Accesso & Tipo                                \\ \hline
		FILM     & E         & 9500    & L                                   \\ \hline
		UTENTE   & E         & 1       & L                                   \\ \hline
		ACCOUNT  & E         & 1       & L                                   \\ \hline
		tipo.usr & R         & 1       & L                                   \\ \hline
		\rowcolor[HTML]{CBCEFB}
		\multicolumn{4}{|l|}{\cellcolor[HTML]{FFCE93}\textbf{Totale}: 9503L} \\ \hline
	\end{tabular}
\end{table}

\subsection{Visualizzare l’elenco dei film in base ad un’età scelta. (Op. 4.2)}
In media gli utenti che richiedono la visualizzazione dei film sono principalmente adulti, quindi la maggior parte visualizzerà tutti i film (9500).
\begin{figure}[H]
	\centering
	\includegraphics{ER/navigazione/visualizzarefilm2.png}
	\caption{Schema di navigazione - Visualizzare elenco film in base all'età}
\end{figure}
\begin{table}[H]
	\centering
	\begin{tabular}{|llll|}
		\hline
		\rowcolor[HTML]{CBCEFB}
		Concetto & Costrutto & Accesso & Tipo                                \\ \hline
		FILM     & E         & 9500    & L                                   \\ \hline
		\rowcolor[HTML]{CBCEFB}
		\multicolumn{4}{|l|}{\cellcolor[HTML]{FFCE93}\textbf{Totale}: 9500L} \\ \hline
	\end{tabular}
\end{table}

\subsection{Visualizzare l’elenco delle serie TV con la relativa durata complessiva e il numero di stagioni ed episodi complessivo, in base all’età di chi lo richiede. (Op. 4.3)} \label{ss:op43}
\begin{figure}[H]
	\centering
	\includegraphics[width=1.2\linewidth]{ER/navigazione/elencoserietv.png}
	\caption{}
	\label{fig:elencoserietv}
\end{figure}
\begin{table}[H]
	\centering
	\begin{tabular}{|llll|}
		\hline
		\rowcolor[HTML]{CBCEFB}
		Concetto & Costrutto & Accesso & Tipo                                   \\ \hline
		UTENTE   & E         & 1       & L                                      \\ \hline
		SERIE    & E         & 2500    & L                                      \\ \hline
		STAGIONE & E         & 10.000  & L                                      \\ \hline
		EPISODIO & E         & 400.000 & L                                      \\ \hline
		\rowcolor[HTML]{CBCEFB}
		\multicolumn{4}{|l|}{\cellcolor[HTML]{FFCE93}\textbf{Totale}: 412.501L} \\ \hline
	\end{tabular}
\end{table}

\subsection{Visualizzare le informazioni riguardanti una serie TV, comprese tutte le stagioni e tutti gli episodi. (Op. 4.4)} \label{ss:op44}
\begin{figure}[H]
	\centering
	\includegraphics[width=1.2\linewidth]{ER/navigazione/infoserietv.png}
	\caption{Schema di navigazione - Visualizzare elenco serie tv in base all'età}
\end{figure}
Per ogni serie vi sono in media $\frac{10000}{2500} = 4$ stagioni, che si compongono ciascuna di $\frac{400000}{10000 * 4} = 10$ episodi complessivi.
\begin{table}[H]
	\centering
	\begin{tabular}{|llll|}
		\hline
		\rowcolor[HTML]{CBCEFB}
		Concetto & Costrutto & Accesso & Tipo                              \\ \hline
		SERIE    & E         & 1       & L                                 \\ \hline
		STAGIONE & E         & 4       & L                                 \\ \hline
		EPISODIO & E         & 40      & L                                 \\ \hline
		\rowcolor[HTML]{CBCEFB}
		\multicolumn{4}{|l|}{\cellcolor[HTML]{FFCE93}\textbf{Totale}: 45L} \\ \hline
	\end{tabular}
\end{table}

\subsection{Contrassegnare come ”visualizzato” un film (Op. 5.1)} \label{ss:op51}
\begin{figure}[H]
	\centering
	\includegraphics[width=300pt]{ER/navigazione/visualizzatofilm.png}
	\caption{Schema di navigazione - Contrassegnare un film "visualizzato"}
\end{figure}
\begin{table}[H]
	\centering
	\begin{tabular}{|llll|}
		\hline
		\rowcolor[HTML]{CBCEFB}
		Concetto            & Costrutto & Accesso & Tipo                  \\ \hline
		visualizzazionefilm & R         & 1       & S                     \\ \hline
		\rowcolor[HTML]{CBCEFB}
		\multicolumn{4}{|l|}{\cellcolor[HTML]{FFCE93}\textbf{Totale}: 1S} \\ \hline
	\end{tabular}
\end{table}

\subsection{Contrassegnare come ”visualizzato” un episodio di una serie (Op. 5.2)} \label{ss:op52}
\begin{figure}[H]
	\centering
	\includegraphics{ER/navigazione/visualizzatoep.png}
	\caption{Schema di navigazione - Contrassegnare un episodio di una serie "visualizzato"}
\end{figure}
\begin{table}[H]
	\centering
	\begin{tabular}{|llll|}
		\hline
		\rowcolor[HTML]{CBCEFB}
		Concetto          & Costrutto & Accesso & Tipo                    \\ \hline
		visualizzazioneep & R         & 1       & S                       \\ \hline
		\rowcolor[HTML]{CBCEFB}
		\multicolumn{4}{|l|}{\cellcolor[HTML]{FFCE93}\textbf{Totale}: 1S} \\ \hline
	\end{tabular}
\end{table}

\subsection{Recensire un film (Op. 6)}
\begin{figure}[H]
	\centering
	\includegraphics[width=450pt]{ER/navigazione/recensionefilm.png}
	\caption{Schema di navigazione - recensire un film}
\end{figure}
\begin{table}[H]
	\centering
	\begin{tabular}{|llll|}
		\hline
		\rowcolor[HTML]{CBCEFB}
		Concetto        & Costrutto & Accesso & Tipo                            \\ \hline
		SEZIONE         & E         & 5       & S                               \\ \hline
		RECFILM         & E         & 1       & S                               \\ \hline
		visual.film     & R         & 1       & L                               \\ \hline
		recfilm.sez     & R         & 5       & S                               \\ \hline
		scritt.rec.film & R         & 1       & S                               \\ \hline
		\rowcolor[HTML]{CBCEFB}
		\multicolumn{4}{|l|}{\cellcolor[HTML]{FFCE93}\textbf{Totale}: 1L + 12S} \\ \hline
	\end{tabular}
\end{table}

\subsection{Visualizzare le recensioni di un film (Op. 7.1)}
Considerando la tabella dei volumi (sez: \ref{s:volumes}), un film ha in media $\frac{950000}{9500} = 100$ recensioni.
\begin{figure}[H]
	\centering
	\includegraphics[width=450pt]{ER/navigazione/visualrecensionifilm.png}
	\caption{Schema di navigazione - Visualizzare le recensioni di un film}
\end{figure}
\begin{table}[H]
	\centering
	\begin{tabular}{|llll|}
		\hline
		\rowcolor[HTML]{CBCEFB}
		Concetto    & Costrutto & Accesso & Tipo                             \\ \hline
		RECFILM     & E         & 100     & L                                \\ \hline
		recfilm.sez & R         & 500     & L                                \\ \hline
		\rowcolor[HTML]{CBCEFB}
		\multicolumn{4}{|l|}{\cellcolor[HTML]{FFCE93}\textbf{Totale}: 600L} \\ \hline
	\end{tabular}
\end{table}

\subsection{Dare una valutazione di utilità ad una recensione di un altro utente su un film. (Op. 7.2)}
\begin{figure}[H]
	\centering
	\includegraphics[width=450pt]{ER/navigazione/valutazionerecfilm.png}
	\caption{Schema di navigazione - Dare una valutazione di utilità ad una recensione di un altro utente su un film.}
\end{figure}
\begin{table}[H]
	\centering
	\begin{tabular}{|llll|}
		\hline
		\rowcolor[HTML]{CBCEFB}
		Concetto     & Costrutto & Accesso & Tipo                         \\ \hline
		val.rec.film & R         & 1       & S                            \\ \hline
		\rowcolor[HTML]{CBCEFB}
		\multicolumn{4}{|l|}{\cellcolor[HTML]{FFCE93}\textbf{Totale}: 1S} \\ \hline
	\end{tabular}
\end{table}

\subsection{Dare una valutazione di utilità ad una recensione di un altro utente su una serie tv. (Op. 7.3)}
\begin{figure}[H]
	\centering
	\includegraphics[width=450pt]{ER/navigazione/valutazionerecserie.png}
	\caption{Schema di navigazione - Dare una valutazione di utilità ad una recensione di un altro utente su una serie.}
\end{figure}
\begin{table}[H]
	\centering
	\begin{tabular}{|llll|}
		\hline
		\rowcolor[HTML]{CBCEFB}
		Concetto      & Costrutto & Accesso & Tipo                        \\ \hline
		val.rec.serie & R         & 1       & S                           \\ \hline
		\rowcolor[HTML]{CBCEFB}
		\multicolumn{4}{|l|}{\cellcolor[HTML]{FFCE93}\textbf{Totale}: 1S} \\ \hline
	\end{tabular}
\end{table}

\subsection{Visualizzare le recensioni di una singola serie (Op. 7.4)}
Considerando la tabella dei volumi (sez: \ref{s:volumes}), una serie ha in media $\frac{250000}{2500} = 100$ recensioni e ciascuna recensione è articolata nel caso con maggior volume in tutte e 5 le sezioni.
\begin{figure}[H]
	\centering
	\includegraphics{ER/navigazione/visualrecserie.png}
	\caption{Schema di navigazione - Visualizzare le recensioni di una serie.}
\end{figure}
\begin{table}[H]
	\centering
	\begin{tabular}{|llll|}
		\hline
		\rowcolor[HTML]{CBCEFB}
		Concetto     & Costrutto & Accesso & Tipo                            \\ \hline
		RECSERIE     & E         & 100     & L                               \\ \hline
		recserie.sez & R         & 500     & L                               \\ \hline
		\rowcolor[HTML]{CBCEFB}
		\multicolumn{4}{|l|}{\cellcolor[HTML]{FFCE93}\textbf{Totale}: 600L} \\ \hline
	\end{tabular}
\end{table}

\subsection{Aggiornare il voto di una sezione di una recensione (Op. 7.5)}
\begin{figure}[H]
	\centering
	\includegraphics{ER/navigazione/visualrecserie.png}
	\caption{Schema di navigazione - Aggiornare una sezione di una recensione.}
\end{figure}
\begin{table}[H]
	\centering
	\begin{tabular}{|llll|}
		\hline
		\rowcolor[HTML]{CBCEFB}
		Concetto     & Costrutto & Accesso & Tipo                            \\ \hline
		recserie.sez     & R         & 1     & S                               \\ \hline
		recserie.sez & R         & 1     & L                               \\ \hline
		\rowcolor[HTML]{CBCEFB}
		\multicolumn{4}{|l|}{\cellcolor[HTML]{FFCE93}\textbf{Totale}: 1S + 1L} \\ \hline
	\end{tabular}
\end{table}

\subsection{Ottenere una classifica dei generi più visualizzati (Op. 8.1)} \label{ss:op81}
Per questa operazione consideriamo che: Ci sono in tutto $13$ generi che vengono memorizzati, questi andranno letti tutti dato che è una classifica. Ci interessano inoltre le relazioni \textit{visualizzazioneep} (Visualizzazione Episodio) e \textit{visualizzazionefilm} (Visualizzazione Film), complessivamente, un utente visualizza in media $\frac{350000}{9960} \approx 35$ film, avendo i film visualizzati, possiamo calcolare quanti generi sono stati visualizzati in media: $\frac{23500}{9500} \approx 2$, ricapitolando i volumi per le visualizzazioni dei film in media sono:
\begin{itemize}
	\item 35 su visualizzazionefilm (per utente).
	\item 2 su cat.film (per film visualizzato).
	\item 13 su genere.
\end{itemize}
Per le serie il ragionamento ha qualche variazione, in particolare per ogni serie vi sono in media \\ $\frac{10000}{2500} = 4$ stagioni,\\ che si compongono ciascuna di \\ $\frac{400000}{10000 * 4} = 10$ episodi complessivi.\\ In media un utente visualizza $\frac{1400000}{9960} \approx 146$ episodi in tutto, quindi si può constatare che un utente visualizza $\frac{146}{10} \approx 15$ stagioni complessive, e di conseguenza $\frac{4}{15} \approx 0,3$ serie.\\
Otteniamo:
\begin{itemize}
	\item 146 su visualizzazioneep (per utente)
	\item 4 su STAGIONE (per utente)
	\item 0,3 su SERIE (per utente)
	\item 2 su cat.serie (per serie visualizzata da un utente)
\end{itemize}
\begin{figure}[H]
	\centering
	\includegraphics[width=450pt]{ER/navigazione/classificageneri.png}
	\caption{Schema di navigazione - visualizzare classifica dei generi più visualizzati}
\end{figure}
\begin{table}[H]
	\centering
	\begin{tabular}{|llll|}
		\hline
		\rowcolor[HTML]{CBCEFB}
		Concetto            & Costrutto & Accesso   & Tipo                         \\ \hline
		GENERE              & E         & 13        & L                            \\ \hline
		STAGIONE            & E         & 39.840    & L                            \\ \hline
		SERIE               & E         & 2.988     & L                            \\ \hline
		visualizzazioneep   & R         & 1.454.160 & L                            \\ \hline
		cat.serie           & R         & 5.976     & L                            \\ \hline
		visualizzazionefilm & R         & 348.600   & L                            \\ \hline
		cat.film            & R         & 697.200   & L                            \\ \hline
		\rowcolor[HTML]{CBCEFB}
		\multicolumn{4}{|l|}{\cellcolor[HTML]{FFCE93}\textbf{Totale}: 2.548.776 L} \\ \hline
	\end{tabular}
\end{table}

\section{Schemi di navigazione e tabelle degli accessi - Operazioni Amministratore}
\subsubsection{Visualizzare una classifica degli utenti con la media valutazioni di utilità peggiore (Op. 9.1)}
Per questa operazione bisogna valutare ogni utente e verificare ogni singola valutazione per le sue recensioni scritte. In media un utente scrive $\frac{1200000}{9960} \approx 120$ recensioni(in particolare $\frac{950000}{9960} \approx 95$ recensioni per i film e $\frac{250000}{9960} \approx 25$ recensioni per le serie tv). \\
Ciascuna recensione per un film riceve $\frac{950000}{115000} \approx 8$ valutazioni mentre per quanto riguarda le recensioni sulle serie tv ciascuna riceve $\frac{250000}{115000} \approx 2$ valutazioni. \\
Quindi complessivamente avremo $95 * 8 = 760$ valutazioni nelle recensioni per i film e $25 * 2 = 50$ valutazioni nelle recensioni per le serie tv che ha svolto un utente, per questo il tutto va moltiplicato per il numero di utenti.

\begin{figure}[H]
	\centering
	\includegraphics[width=450pt]{ER/navigazione/classificautilitápeggiore.png}
	\caption{Schema di navigazione - visualizzare classifica degli utenti con la media valutazioni di utilità peggiore}
\end{figure}
\begin{table}[H]
	\centering
	\begin{tabular}{|llll|}
		\hline
		\rowcolor[HTML]{CBCEFB}
		Concetto      & Costrutto & Accesso   & Tipo                               \\ \hline
		UTENTE        & E         & 9960      & L                                  \\ \hline
		val.rec.film  & R         & 7.569.600 & L                                  \\ \hline
		val.rec.serie & R         & 498.000   & L                                  \\ \hline
		\rowcolor[HTML]{CBCEFB}
		\multicolumn{4}{|l|}{\cellcolor[HTML]{FFCE93}\textbf{Totale}: 8.077.560 L} \\ \hline
	\end{tabular}
\end{table}

\subsubsection{Visualizzare una classifica degli utenti con la media valutazioni di utilità migliori (Op. 9.2)}
Per questa operazione il ragionamento è analogo a quella precedente(Op. 9.1).
\begin{figure}[H]
	\centering
	\includegraphics[width=450pt]{ER/navigazione/classificautilitámigliore.png}
	\caption{Schema di navigazione - visualizzare classifica degli utenti con la media valutazioni di utilità migliori}
\end{figure}
\begin{table}[H]
	\centering
	\begin{tabular}{|llll|}
		\hline
		\rowcolor[HTML]{CBCEFB}
		Concetto      & Costrutto & Accesso   & Tipo                               \\ \hline
		UTENTE        & E         & 9960      & L                                  \\ \hline
		val.rec.film  & R         & 7.569.600 & L                                  \\ \hline
		val.rec.serie & R         & 498.000   & L                                  \\ \hline
		\rowcolor[HTML]{CBCEFB}
		\multicolumn{4}{|l|}{\cellcolor[HTML]{FFCE93}\textbf{Totale}: 8.077.560 L} \\ \hline
	\end{tabular}
\end{table}

\subsubsection{Assegnamento in blocco di promozioni ai primi 5 utenti tesserati con la media valutazioni di utilità migliori (Op. 10)}
\begin{figure}[H]
	\centering
	\includegraphics[width=450pt]{ER/navigazione/couponutentimigliori.png}
	\caption{Schema di navigazione - Operazione 10}
\end{figure}
\begin{table}[H]
	\centering
	\begin{tabular}{|llll|}
		\hline
		\rowcolor[HTML]{CBCEFB}
		Concetto     & Costrutto & Accesso & Tipo                         \\ \hline
		premitessera & R         & 5       & S                            \\ \hline
		\rowcolor[HTML]{CBCEFB}
		\multicolumn{4}{|l|}{\cellcolor[HTML]{FFCE93}\textbf{Totale}: 5S} \\ \hline
	\end{tabular}
\end{table}


\subsubsection{Aggiunta di un nuovo film alla piattaforma (Op. 11.1)}
\begin{figure}[H]
	\centering
	\includegraphics[width=150pt]{ER/navigazione/aggiuntafilm.png}
	\caption{Schema di navigazione - aggiunta nuovo film alla piattaforma}
\end{figure}
\begin{table}[H]
	\centering
	\begin{tabular}{|llll|}
		\hline
		\rowcolor[HTML]{CBCEFB}
		Concetto & Costrutto & Accesso & Tipo                             \\ \hline
		FILM     & E         & 1       & S                                \\ \hline
		\rowcolor[HTML]{CBCEFB}
		\multicolumn{4}{|l|}{\cellcolor[HTML]{FFCE93}\textbf{Totale}: 1S} \\ \hline
	\end{tabular}
\end{table}

\subsubsection{Aggiunta di persone che hanno realizzato un film alla piattaforma (Op. 11.2)}
\begin{figure}[H]
	\centering
	\includegraphics[width=450pt]{ER/navigazione/aggiuntacast.png}
	\caption{Schema di navigazione - aggiunta di persone che hanno realizzato un film}
\end{figure}
\begin{table}[H]
	\centering
	\begin{tabular}{|llll|}
		\hline
		\rowcolor[HTML]{CBCEFB}
		Concetto    & Costrutto & Accesso & Tipo                          \\ \hline
		MEMBROCAST  & E         & 1       & S                             \\ \hline
		membro.cast & R         & 1       & S                             \\ \hline
		\rowcolor[HTML]{CBCEFB}
		\multicolumn{4}{|l|}{\cellcolor[HTML]{FFCE93}\textbf{Totale}: 2S} \\ \hline
	\end{tabular}
\end{table}

\subsection{Aggiunta di una nuova stagione per una specifica serie TV. (Op. 11.3)} \label{ss:op45}
Per questa Operazione, é necessaria la lettura della serie per poter aggiungere una nuova stagione e legarla tramite l'associazione compSerie. Inoltre, va aggiunto un nuovo episodio alla stagione aggiunta, essendo la cardinalitá minima obbligatoria che lega episodio a stagione.
\begin{figure}[H]
	\centering
	\includegraphics[width=1.2\linewidth]{ER/navigazione/aggiuntastagione.png}
	\caption{Schema di navigazione - Aggiunta di nuova stagione per serie TV}
\end{figure}
\begin{table}[H]
	\centering
	\begin{tabular}{|llll|}
		\hline
		\rowcolor[HTML]{CBCEFB}
		Concetto 		& Costrutto & Accesso 	& Tipo                              \\ \hline
		SERIE   		& E     	& 1       	& L                                 \\ \hline
		STAGIONE 		& E         & 1       	& S                                 \\ \hline
		EPISODIO 		& E         & 1       	& S                                 \\ \hline
		compSerie 		& R         & 1       	& S                                 \\ \hline
		compStagione	& R         & 1       	& S                                 \\ \hline
		\rowcolor[HTML]{CBCEFB}
		\multicolumn{4}{|l|}{\cellcolor[HTML]{FFCE93}\textbf{Totale}: 1L+4S} \\ \hline
	\end{tabular}
\end{table}

\subsection{Aggiunta di un nuovo episodio per una specifica stagione di una serie TV. (Op. 11.4)} \label{ss:op46}
\begin{figure}[H]
	\centering
	\includegraphics[width=1.2\linewidth]{ER/navigazione/aggiuntaepisodio.png}
	\caption{Schema di navigazione - Aggiunta di nuovo episodio per stagione serie TV.}
\end{figure}
\begin{table}[H]
	\centering
	\begin{tabular}{|llll|}
		\hline
		\rowcolor[HTML]{CBCEFB}
		Concetto 		& Costrutto & Accesso 	& Tipo                              \\ \hline
		STAGIONE 		& E         & 1       	& L                                 \\ \hline
		EPISODIO 		& E         & 1       	& S                                 \\ \hline
		compStagione	& R         & 1       	& S                                 \\ \hline
		\rowcolor[HTML]{CBCEFB}
		\multicolumn{4}{|l|}{\cellcolor[HTML]{FFCE93}\textbf{Totale}: 1L+2S} \\ \hline
	\end{tabular}
\end{table}

\subsubsection{Registrazione di una nuova tessera (Op. 12)}
\begin{figure}[H]
	\centering
	\includegraphics[width=450pt]{ER/navigazione/regtessera.png}
	\caption{Schema di navigazione - Registrazione di una tessera}
\end{figure}
\begin{table}[H]
	\centering
	\begin{tabular}{|llll|}
		\hline
		\rowcolor[HTML]{CBCEFB}
		Concetto     & Costrutto & Accesso & Tipo                         \\ \hline
		TESSERA      & E         & 1       & S                            \\ \hline
		appartenenza & R         & 1       & S                            \\ \hline
		tesseramento & R         & 1       & S                            \\ \hline
		\rowcolor[HTML]{CBCEFB}
		\multicolumn{4}{|l|}{\cellcolor[HTML]{FFCE93}\textbf{Totale}: 3S} \\ \hline
	\end{tabular}
\end{table}


\section{Analisi delle ridondanze}
Le ridondanze principali che vogliamo analizzare sono le seguenti:
\begin{enumerate}
	\item Attributo "NumeroVisualizzati" nell'entità GENERE.
	\item Attributo "DurataComplessiva" nell'entità SERIE.
	\item Attributo "NumeroEpisodi" nell'entità STAGIONE.
	\item Attribito "NumeroStagioni" nell'entità SERIE.
	\item Attributo "VotoComplessivo" nell'entità RECENSIONE.
\end{enumerate}
\subsection{Attributo 1: NumeroVisualizzati}
\subsubsection{Operazioni interessate:}
\begin{itemize}
	\item Op. 8.1: Ottenere una classifica dei generi più visualizzati
	\item Op. 5.(1-2): Contrassegnare come "visualizzato" un film o un episodio di una serie.
\end{itemize}
\subsubsection{Tabelle degli accessi senza ridondanza:}
\begin{itemize}
	\item Per operazione 8.1: Vedi la sottosezione \ref{ss:op81}, essa ha frequenza $5/gg$ dandoci complessivamente 12.743.880 accessi in lettura al giorno.
	\item Per operazione 5.1: Vedi la sottosezione \ref{ss:op51}, essa ha frequenza $350/gg$ dandoci complessivamente 350 accessi in scrittura al giorno.
	\item Per operazione 5.2: Vedi la sottosezione \ref{ss:op52}, essa ha frequenza $500/gg$ dandoci complessivamente 500 accessi in scrittura al giorno.
\end{itemize}
Complessivamente quindi, abbiamo $2 * (500 + 350) + 12.743.880 = 12.745.580$ accessi complessivi.
\subsubsection{Tabelle degli accessi con ridondanza:}
Consideriamo il caso in cui si introduca un attributo ridondante "NumeroVisualizzazioni" sull'entità "GENERE", che viene incrementato ogni volta che un utente contrassegna come visualizzato un film, o tutti gli episodi correnti di una serie.
\begin{itemize}
	\item Per operazione 5.1: Per effettuare questa operazione, oltre che 1S per scrivere la relazione \textit{visualizzazionefilm} è necessario un aggiornamento su genere per incrementare il valore dell'attributo ridondante, avendo una complessità complessiva di 1L + 2S, considerando la frequenza di $350/gg$ questa ci crea $350 + 4 * 350 = 1750$ accessi complessivi al giorno.
	\item Per operazione 5.2: Questa invece ci obbliga a calcolare quanti episodi servono per poter considerare una serie come visualizzata, sappiamo che una serie è composta in media da 4 stagioni da 10 episodi ciascuna, quindi se vengono contrassegnati 500 episodi al giorno, significa che vengono visualizzate in media $\frac{500}{4 * 10} = 12,5$ serie al giorno, comportando 12,5S + 12,5L + 500S accessi, dando una complessità complessiva di $3 * 12,5 + 1000 = 1037,5$ al giorno.
	\item Per operazione 8.1: Con l'attributo ridondante questa operazione si limita alla lettura dell'attributo "NumeroVisualizzati" nell'entità "GENERE", considerando che ce ne sono 13 si hanno 13L con una frequenza di 5 al giorno, quindi complessivamente $13 * 5$ letture al giorno dando una complessità di 65 accessi.
\end{itemize}
In totale si ottiene da questo studio $65 + 1037,5 + 1750 = 2852.5$ accessi complessivi.
\subsubsection{Conclusione:}
Considerando i valori ottenuti dall'analisi precedente, si può affermare che data la complessità $2852.5 \ll 12.745.580$ è di gran lunga conveniente introdurre l'attributo ridondante "NumeroVisualizzati" in GENERE, privilegiando l'efficienza rispetto all'occupazione di spazio.

\subsection{Attributo 2: DurataComplessiva}
\subsubsection{Operazioni interessate:}
\begin{itemize}
	\item Op. 4.3: Visualizzare l’elenco delle serie TV con la relativa durata complessiva e il numero di stagioni ed episodi complessivo, in base all’età di chi lo richiede.
	\item Op. 4.4: Visualizzare le informazioni riguardanti una serie TV, comprese tutte le stagioni e tutti gli episodi.
	\item Op. 4.5: Aggiunta di una nuova stagione per una specifica serie TV.
	\item Op. 4.6: Aggiunta di un nuovo episodio per una specifica stagione di una serie TV.
\end{itemize}
\subsubsection{Tabelle degli accessi senza ridondanza:}
\begin{itemize}
	\item Per operazione 4.3: Vedi la sottosezione \ref{ss:op43}, essa ha frequenza $100/gg$ dandoci complessivamente accessi 41.250.100 in lettura al giorno.
	\item Per operazione 4.4: Vedi la sottosezione \ref{ss:op44}, essa ha frequenza $50/gg$ dandoci complessivamente accessi 2.250 in lettura al giorno.
	\item Per operazione 4.5: Vedi la sottosezione \ref{ss:op45}, essa ha frequenza $5/gg$ dandoci complessivamente accessi 5 in lettura e 40 in scrittura al giorno.
	\item Per operazione 4.6: Vedi la sottosezione \ref{ss:op46}, essa ha frequenza $10/gg$ dandoci complessivamente accessi 10 in lettura e 40 in scrittura al giorno.
\end{itemize}
Complessivamente quindi, abbiamo $41.250.100 + 2.250 + 45 + 50 = 41.252.445$ accessi complessivi.
\subsubsection{Tabelle degli accessi con ridondanza:}
Consideriamo il caso in cui si introduca il seguente attributo ridondante "DurataComplessiva" sull'entità "SERIE", questo attributo serve per dichiarare la durata complessiva(quindi la sommatoria della durata di ciascun episodio) di una serie tv.
\begin{itemize}
	\item Per operazione 4.3: Quest'operazione è analoga al caso senza ridondanza.
	\item Per operazione 4.4: Quest'operazione è analoga al caso senza ridondanza.
	\item Per operazione 4.5: Per questa operazione gli accessi in scrittura aumentano dato che bisogna fare un aggiornamento nell'entità SERIE per incrementare l'attributo ridondante "DurataComplessiva" a seguito dell'aggiunta in STAGIONE ed EPISODIO. Quindi avremo un totale di 1L + 5S che messe in relazione con la frequenza giornaliera darà come risultato un totale di 5 accessi in lettura e 50 accessi in scrittura al giorno.
	\item Per operazione 4.6: Anche in quest'operazione bisogna aumentare la durata complessiva nell'entità SERIE quindi effettuare un aggiornamento ottenendo in totale 2L + 3S che messe in relazione con la frequenza giornaliera darà come risultato un totale di 20 accessi in lettura e 60 accessi in scrittura al giorno.
\end{itemize}

In totale si ottiene da questo studio $41.250.100 + 2.250 + 55 + 80 = 41.252.485$ accessi complessivi.
\subsubsection{Conclusioni:}
Considerando i valori ottenuti dall'analisi precedente, possiamo affermare che data la complessità $41.252.485 > 41.252.445$ NON conviene introdurre l'attributo ridondante "DurataComplessiva" in SERIE (perlomeno non singolarmente).

\subsection{Attributo 3: NumeroEpisodi}
\subsubsection{Operazioni interessate:}
\begin{itemize}
	\item Op. 4.3: Visualizzare l’elenco delle serie TV con la relativa durata complessiva e il numero di stagioni ed episodi complessivo, in base all’età di chi lo richiede.
	\item Op. 4.4: Visualizzare le informazioni riguardanti una serie TV, comprese tutte le stagioni e tutti gli episodi.
	\item Op. 4.5: Aggiunta di una nuova stagione per una specifica serie TV.
	\item Op. 4.6: Aggiunta di un nuovo episodio per una specifica stagione di una serie TV.
\end{itemize}
\subsubsection{Tabelle degli accessi senza ridondanza:}
\begin{itemize}
	\item Per operazione 4.3: Vedi la sottosezione \ref{ss:op43}, essa ha frequenza $100/gg$ dandoci complessivamente accessi 41.250.100 in lettura al giorno.
	\item Per operazione 4.4: Vedi la sottosezione \ref{ss:op44}, essa ha frequenza $50/gg$ dandoci complessivamente accessi 2.250 in lettura al giorno.
	\item Per operazione 4.5: Vedi la sottosezione \ref{ss:op45}, essa ha frequenza $5/gg$ dandoci complessivamente accessi 5 in lettura e 40 in scrittura al giorno.
	\item Per operazione 4.6: Vedi la sottosezione \ref{ss:op46}, essa ha frequenza $10/gg$ dandoci complessivamente accessi 10 in lettura e 40 in scrittura al giorno.
\end{itemize}
Complessivamente quindi, abbiamo $41.250.100 + 2.250 + 45 + 50 = 41.252.445$ accessi complessivi.
\subsubsection{Tabelle degli accessi con ridondanza:}
Consideriamo il caso in cui si introduca il seguente attributo ridondante "NumeroEpisodi" sull'entità "STAGIONE", questo attributo serve per dichiarare la quantità di episodi che è composta una STAGIONE di una specifica SERIE.
\begin{itemize}
	\item Per operazione 4.3: Quest'operazione è analoga al caso senza ridondanza.
	\item Per operazione 4.4: Quest'operazione è analoga al caso senza ridondanza.
	\item Per operazione 4.5: Quest'operazione è analoga al caso senza ridondanza.
	\item Per operazione 4.6: In quest'operazione bisogna incrementare "NumeroEpisodi" nell'entità STAGIONE quindi effettuare un aggiornamento ottenendo in totale 1L + 3S che messe in relazione con la frequenza giornaliera darà come risultato un totale di 10 accessi in lettura e 60 accessi in scrittura al giorno.
\end{itemize}

In totale si ottiene da questo studio $41.250.100 + 2.250 + 45 + 70 = 41.252.465$ accessi complessivi.
\subsubsection{Conclusioni:}
Considerando i valori ottenuti dall'analisi precedente, possiamo affermare che data la complessità $41.252.465 > 41.252.445$ NON conviene introdurre l'attributo ridondante "NumeroEpisodi" in STAGIONE (perlomeno non singolarmente).


\subsection{Attributo 4: NumeroStagioni}
\subsubsection{Operazioni interessate:}
\begin{itemize}
	\item Op. 4.3: Visualizzare l’elenco delle serie TV con la relativa durata complessiva e il numero di stagioni ed episodi complessivo, in base all’età di chi lo richiede.
	\item Op. 4.4: Visualizzare le informazioni riguardanti una serie TV, comprese tutte le stagioni e tutti gli episodi.
	\item Op. 4.5: Aggiunta di una nuova stagione per una specifica serie TV.
	\item Op. 4.6: Aggiunta di un nuovo episodio per una specifica stagione di una serie TV.
\end{itemize}
\subsubsection{Tabelle degli accessi senza ridondanza:}
\begin{itemize}
	\item Per operazione 4.3: Vedi la sottosezione \ref{ss:op43}, essa ha frequenza $100/gg$ dandoci complessivamente accessi 41.250.100 in lettura al giorno.
	\item Per operazione 4.4: Vedi la sottosezione \ref{ss:op44}, essa ha frequenza $50/gg$ dandoci complessivamente accessi 2.250 in lettura al giorno.
	\item Per operazione 4.5: Vedi la sottosezione \ref{ss:op45}, essa ha frequenza $5/gg$ dandoci complessivamente accessi 5 in lettura e 40 in scrittura al giorno.
	\item Per operazione 4.6: Vedi la sottosezione \ref{ss:op46}, essa ha frequenza $10/gg$ dandoci complessivamente accessi 10 in lettura e 40 in scrittura al giorno.
\end{itemize}
Complessivamente quindi, abbiamo $ 41.250.100 + 2.250 + 45 + 50 = 41.252.445$ accessi complessivi.
\subsubsection{Tabelle degli accessi con ridondanza:}
Consideriamo il caso in cui si introduca il seguente attributo ridondante "NumeroStagioni" sull'entità "SERIE", questo attributo serve per dichiarare la quantità di stagioni che è composta una SERIE.
\begin{itemize}
	\item Per operazione 4.3: Quest'operazione è analoga al caso senza ridondanza.
	\item Per operazione 4.4: Quest'operazione è analoga al caso senza ridondanza.
	\item Per operazione 4.5: Per questa operazione è necessario l'incremento dell'attributo ridondante in SERIE, quindi questo aggiornamento porta ad avere gli accessi totali a 1L in lettura e 5S. Mettendo questi accessi in relazione con la frequenza, otteniamo in totale 5 accessi in lettura e 100 accessi in scrittura al giorno.
	\item Per operazione 4.6: Quest'operazione è analoga al caso senza ridondanza.
\end{itemize}

In totale si ottiene da questo studio $41.250.100 + 2.250 + 105 + 50 = 41.252.505$ accessi complessivi.
\subsubsection{Conclusioni:}
Considerando i valori ottenuti dall'analisi precedente, possiamo affermare che data la complessità $41.252.505 > 41.252.445$ NON conviene introdurre l'attributo ridondante "NumeroStagioni" in SERIE (perlomeno non singolarmente).


\subsection{Attributo 2-3: DurataComplessiva e NumeroEpisodi}
\subsubsection{Operazioni interessate:}
\begin{itemize}
	\item Op. 4.3: Visualizzare l’elenco delle serie TV con la relativa durata complessiva e il numero di stagioni ed episodi complessivo, in base all’età di chi lo richiede.
	\item Op. 4.4: Visualizzare le informazioni riguardanti una serie TV, comprese tutte le stagioni e tutti gli episodi.
	\item Op. 4.5: Aggiunta di una nuova stagione per una specifica serie TV.
	\item Op. 4.6: Aggiunta di un nuovo episodio per una specifica stagione di una serie TV.
\end{itemize}
\subsubsection{Tabelle degli accessi senza ridondanza:}
\begin{itemize}
	\item Per operazione 4.3: Vedi la sottosezione \ref{ss:op43}, essa ha frequenza $100/gg$ dandoci complessivamente accessi 41.250.100 in lettura al giorno.
	\item Per operazione 4.4: Vedi la sottosezione \ref{ss:op44}, essa ha frequenza $50/gg$ dandoci complessivamente accessi 2.250 in lettura al giorno.
	\item Per operazione 4.5: Vedi la sottosezione \ref{ss:op45}, essa ha frequenza $5/gg$ dandoci complessivamente accessi 5 in lettura e 40 in scrittura al giorno.
	\item Per operazione 4.6: Vedi la sottosezione \ref{ss:op46}, essa ha frequenza $10/gg$ dandoci complessivamente accessi 10 in lettura e 40 in scrittura al giorno.
\end{itemize}
Complessivamente quindi, abbiamo $41.250.100 + 2.250 + 45 + 50 = 41.252.445$ accessi complessivi.
\subsubsection{Tabelle degli accessi con ridondanza:}
Consideriamo il caso in cui si introducano i seguenti attributi ridondanti "DurataComplessiva" e "NumeroEpisodi" relativamente sulle entità "SERIE" e "STAGIONE", questi attributi servono per sapere la durata totale(in minuti) di una serie e il numero di episodi per una stagione di una serie TV.
\begin{itemize}
	\item Per operazione 4.3: Quest'operazione è analoga al caso senza ridondanza se non per il fatto che non è più necessario leggere ciascun episodio, grazie al fatto che manteniamo salvato il numero di episodi tramite l'attributo ridondante in STAGIONE. Quindi avremo in totale 12.501L, che messe in relazione con la frequenza diventano 1.250.100 accessi in lettura al giorno.
	\item Per operazione 4.4: Quest'operazione è analoga al caso senza ridondanza.
	\item Per operazione 4.5: Per questa operazione gli accessi in scrittura aumentano dato che bisogna fare un aggiornamento nell'entità SERIE per incrementare l'attributo ridondante "DurataComplessiva" ed anche nell'entità STAGIONE per incrementare l'attributo ridondante "NumeroEpisodi" a seguito dell'aggiunta di una STAGIONE con il relativo EPISODIO (aggiunto obbligatoriamente, data la cardinalità). 
	Quindi avremo un totale di 2L + 5S che messe in relazione con la frequenza giornaliera darà come risultato un totale di 10 accessi in lettura e 50 accessi in scrittura al giorno.
	\item Per operazione 4.6: Per questa operazione il ragionamento è analogo alla precedente. Quindi otterremo un totale di 2L + 3S che messe in relazione con la frequenza giornaliera darà come risultato un totale di 20 accessi in lettura e 60 accessi in scrittura al giorno.
\end{itemize}

In totale si ottiene da questo studio $1.250.100 + 2.250 + 60 + 80 = 1.252.490$ accessi complessivi.
\subsubsection{Conclusioni:}
Considerando i valori ottenuti dall'analisi precedente, possiamo affermare che data la complessità $1.252.490 \ll 41.252.445$ conviene introdurre, in coppia, gli attributi ridondanti "DurataComplessiva" e "NumeroEpisodi" relativamente nelle entità SERIE e STAGIONE.


\subsection{Attributo 2-4: DurataComplessiva e NumeroStagioni}
\subsubsection{Operazioni interessate:}
\begin{itemize}
	\item Op. 4.3: Visualizzare l’elenco delle serie TV con la relativa durata complessiva e il numero di stagioni ed episodi complessivo, in base all’età di chi lo richiede.
	\item Op. 4.4: Visualizzare le informazioni riguardanti una serie TV, comprese tutte le stagioni e tutti gli episodi.
	\item Op. 4.5: Aggiunta di una nuova stagione per una specifica serie TV.
	\item Op. 4.6: Aggiunta di un nuovo episodio per una specifica stagione di una serie TV.
\end{itemize}
\subsubsection{Tabelle degli accessi senza ridondanza:}
\begin{itemize}
	\item Per operazione 4.3: Vedi la sottosezione \ref{ss:op43}, essa ha frequenza $100/gg$ dandoci complessivamente accessi 41.250.100 in lettura al giorno.
	\item Per operazione 4.4: Vedi la sottosezione \ref{ss:op44}, essa ha frequenza $50/gg$ dandoci complessivamente accessi 2.250 in lettura al giorno.
	\item Per operazione 4.5: Vedi la sottosezione \ref{ss:op45}, essa ha frequenza $5/gg$ dandoci complessivamente accessi 5 in lettura e 40 in scrittura al giorno.
	\item Per operazione 4.6: Vedi la sottosezione \ref{ss:op46}, essa ha frequenza $10/gg$ dandoci complessivamente accessi 10 in lettura e 40 in scrittura al giorno.
\end{itemize}
Complessivamente quindi, abbiamo $41.250.100 + 2.250 + 45 + 50 = 41.252.445$ accessi complessivi.
\subsubsection{Tabelle degli accessi con ridondanza:}
Consideriamo il caso in cui si introducano i seguenti attributi ridondanti "DurataComplessiva" e "NumeroStagioni" relativamente sulla entità "SERIE", questi attributi servono per sapere la durata totale(in minuti) di una serie e il numero di stagioni per una una serie TV.
\begin{itemize}
	\item Per operazione 4.3: Quest'operazione è analoga al caso senza ridondanza se non per il fatto che non è più necessario leggere ciascuna stagione, grazie al fatto che manteniamo salvato il numero di stagioni tramite l'attributo ridondante in SERIE. Quindi avremo in totale 402.501L, che messe in relazione con la frequenza diventano 40.250.100 accessi in lettura al giorno.
	\item Per operazione 4.4: Quest'operazione è analoga al caso senza ridondanza.
	\item Per operazione 4.5: Per questa operazione gli accessi in scrittura aumentano dato che bisogna fare un aggiornamento nell'entità SERIE per incrementare l'attributo ridondante "DurataComplessiva" e "NumeroStagioni" a seguito dell'aggiunta di una STAGIONE con il relativo EPISODIO (aggiunto obbligatoriamente, data la cardinalità). 
	Quindi avremo un totale di 1L + 5S che messe in relazione con la frequenza giornaliera darà come risultato un totale di 5 accessi in lettura e 50 accessi in scrittura al giorno.
	\item Per operazione 4.6: Per questa operazione il ragionamento è analogo alla precedente. Quindi otterremo un totale di 2L + 3S, aggiornando "DurataComplessiva" di SERIE, che messe in relazione con la frequenza giornaliera darà come risultato un totale di 20 accessi in lettura e 60 accessi in scrittura al giorno.
\end{itemize}

In totale si ottiene da questo studio $40.250.100 + 2.250 + 55 + 80 = 40.252.485$ accessi complessivi.
\subsubsection{Conclusioni:}
Considerando i valori ottenuti dall'analisi precedente, possiamo affermare che data la complessità $40.252.485 < 41.252.445$ conviene introdurre, in coppia, gli attributi ridondanti "DurataComplessiva" e "NumeroStagioni" relativamente nella entità SERIE.


\subsection{Attributo 3-4: NumeroEpisodi e NumeroStagioni}
\subsubsection{Operazioni interessate:}
\begin{itemize}
	\item Op. 4.3: Visualizzare l’elenco delle serie TV con la relativa durata complessiva e il numero di stagioni ed episodi complessivo, in base all’età di chi lo richiede.
	\item Op. 4.4: Visualizzare le informazioni riguardanti una serie TV, comprese tutte le stagioni e tutti gli episodi.
	\item Op. 4.5: Aggiunta di una nuova stagione per una specifica serie TV.
	\item Op. 4.6: Aggiunta di un nuovo episodio per una specifica stagione di una serie TV.
\end{itemize}
\subsubsection{Tabelle degli accessi senza ridondanza:}
\begin{itemize}
	\item Per operazione 4.3: Vedi la sottosezione \ref{ss:op43}, essa ha frequenza $100/gg$ dandoci complessivamente accessi 41.250.100 in lettura al giorno.
	\item Per operazione 4.4: Vedi la sottosezione \ref{ss:op44}, essa ha frequenza $50/gg$ dandoci complessivamente accessi 2.250 in lettura al giorno.
	\item Per operazione 4.5: Vedi la sottosezione \ref{ss:op45}, essa ha frequenza $5/gg$ dandoci complessivamente accessi 5 in lettura e 40 in scrittura al giorno.
	\item Per operazione 4.6: Vedi la sottosezione \ref{ss:op46}, essa ha frequenza $10/gg$ dandoci complessivamente accessi 10 in lettura e 40 in scrittura al giorno.
\end{itemize}
Complessivamente quindi, abbiamo $41.250.100 + 2.250 + 45 + 50 = 41.252.445$ accessi complessivi.
\subsubsection{Tabelle degli accessi con ridondanza:}
Consideriamo il caso in cui si introducano i seguenti attributi ridondanti "NumeroEpisodi" e "NumeroStagioni" relativamente sulle entità "STAGIONE" e "SERIE", questi attributi servono per sapere il numero di episodi di una serie e il numero di stagioni per una serie TV.
\begin{itemize}
	\item Per operazione 4.3: Quest'operazione è analoga al caso senza ridondanza.
	\item Per operazione 4.4: Quest'operazione è analoga al caso senza ridondanza.
	\item Per operazione 4.5: Per questa operazione gli accessi in scrittura aumentano dato che bisogna fare un aggiornamento nell'entità SERIE per incrementare l'attributo ridondante "NumeroStagioni" a seguito dell'aggiunta di un'entità STAGIONE con il relativo EPISODIO (aggiunto obbligatoriamente, data la cardinalità). 
	Quindi avremo un totale di 1L + 5S che messe in relazione con la frequenza giornaliera darà come risultato un totale di 5 accessi in lettura e 50 accessi in scrittura al giorno.
	\item Per operazione 4.6: Per questa operazione il ragionamento è analogo alla precedente. Quindi otterremo un totale di 1L + 3S, che messe in relazione con la frequenza giornaliera darà come risultato un totale di 10 accessi in lettura e 60 accessi in scrittura al giorno.
\end{itemize}

In totale si ottiene da questo studio $41.250.100 + 2.250 + 55 + 70 = 41.252.475$ accessi complessivi.
\subsubsection{Conclusioni:}
Considerando i valori ottenuti dall'analisi precedente, possiamo affermare che data la complessità $41.252.475 > 41.252.445$ NON conviene introdurre, in coppia, gli attributi ridondanti "NumeroEpisodi" e "NumeroStagioni" relativamente nelle entità SERIE e STAGIONE.


\subsection{Attributo 2-3-4: DurataComplessiva, NumeroEpisodi e NumeroStagioni}
\subsubsection{Operazioni interessate:}
\begin{itemize}
	\item Op. 4.3: Visualizzare l’elenco delle serie TV con la relativa durata complessiva e il numero di stagioni ed episodi complessivo, in base all’età di chi lo richiede.
	\item Op. 4.4: Visualizzare le informazioni riguardanti una serie TV, comprese tutte le stagioni e tutti gli episodi.
	\item Op. 4.5: Aggiunta di una nuova stagione per una specifica serie TV.
	\item Op. 4.6: Aggiunta di un nuovo episodio per una specifica stagione di una serie TV.
\end{itemize}
\subsubsection{Tabelle degli accessi senza ridondanza:}
\begin{itemize}
	\item Per operazione 4.3: Vedi la sottosezione \ref{ss:op43}, essa ha frequenza $100/gg$ dandoci complessivamente accessi 41.250.100 in lettura al giorno.
	\item Per operazione 4.4: Vedi la sottosezione \ref{ss:op44}, essa ha frequenza $50/gg$ dandoci complessivamente accessi 2.250 in lettura al giorno.
	\item Per operazione 4.5: Vedi la sottosezione \ref{ss:op45}, essa ha frequenza $5/gg$ dandoci complessivamente accessi 5 in lettura e 40 in scrittura al giorno.
	\item Per operazione 4.6: Vedi la sottosezione \ref{ss:op46}, essa ha frequenza $10/gg$ dandoci complessivamente accessi 10 in lettura e 40 in scrittura al giorno.
\end{itemize}
Complessivamente quindi, abbiamo $41.250.100 + 2.250 + 45 + 50 = 41.252.445$ accessi complessivi.
\subsubsection{Tabelle degli accessi con ridondanza:}
Consideriamo il caso in cui si introducano i seguenti attributi ridondanti "DurataComplessiva", "NumeroStagioni" e "NumeroEpisodi" sulle entità "SERIE" e "STAGIONE", questi attributi servono per sapere la durata complessiva di ciascun episodio, il numero di episodi di una serie e il numero di stagioni per una serie TV.
\begin{itemize}
	\item Per operazione 4.3: Quest'operazione viene ridotta notevolmente visto che abbiamo tutte le informazioni necessarie all'interno dell'entità SERIE e STAGIONE. Quindi in totale saranno fatti 12500L che messe in relazione con la frequenza giornaliera darà come risultato un totale di 1.250.000 accessi in lettura al giorno.
	\item Per operazione 4.4: Quest'operazione è analoga al caso senza ridondanza.
	\item Per operazione 4.5: Per questa operazione gli accessi in scrittura aumentano dato che bisogna fare un aggiornamento nell'entità SERIE per incrementare gli attributi ridondante "NumeroStagioni" e "DurataComplessiva" a seguito dell'aggiunta di un'entità STAGIONE con il relativo EPISODIO (aggiunto obbligatoriamente, data la cardinalità). 
	Quindi avremo un totale di 1L + 5S che messe in relazione con la frequenza giornaliera darà come risultato un totale di 5 accessi in lettura e 50 accessi in scrittura al giorno.
	\item Per operazione 4.6: Per questa operazione il ragionamento è analogo alla precedente. Quindi otterremo un totale di 2L + 4S, che messe in relazione con la frequenza giornaliera darà come risultato un totale di 20 accessi in lettura e 80 accessi in scrittura al giorno.
\end{itemize}

In totale si ottiene da questo studio $1.250.000 + 2.250 + 55 + 100 = 1.252.405$ accessi complessivi.
\subsubsection{Conclusioni:}
Considerando i valori ottenuti dall'analisi precedente, possiamo affermare che data la complessità $1.252.405 \ll 41.252.445$ conviene introdurre, in trio, gli attributi ridondanti "DurataComplessiva", "NumeroStagioni" e "NumeroEpisodi" relativamente nelle entità SERIE e STAGIONE.


\subsection{Attributo 5: VotoComplessivo}
\subsubsection{Operazioni interessate:}
\begin{itemize}
	\item Op. 7.1: Visualizzare le recensioni di un film.
	\item Op. 7.4: Visualizzare le recensioni di una serie.
\end{itemize}
\subsubsection{Tabelle degli accessi senza ridondanza:}
\begin{itemize}
	\item Per operazione 7.1: 600L con una frequenza di $250/gg$.
	\item Per operazione 7.4: 600L con una frequenza di $150/gg$.
\end{itemize}
In tutto abbiamo $600 * 250 + 600 * 150 = 252.500$ accessi complessivi.
\subsubsection{Tabelle degli accessi con ridondanza:}
Supponiamo di introdurre un "VotoComplessivo" su ogni singola recensione, le letture non diminuiscono dato che comunque vanno reperiti i dati sulle diverse sezioni per poterle visualizzare:
\begin{itemize}
	\item Op. 7.1: 600L con una frequenza di $250/gg$.
	\item Op. 7.4: 600L con una frequenza di $150/gg$.
	\item Op. 7.5: 1S + 1L + 5L (il ricalcolo del voto complessivo) con una frequenza di $50/gg$
\end{itemize}
\subsubsection{Conclusione:}
Si nota che nel secondo caso gli accessi sono maggiori rispetto al primo, quindi l'introduzione dell'attributo ridondante non è conveniente.

\subsection{Conclusioni finale sull'analisi delle ridondanze}
A seguito delle analisi svolte sugli attributi che causavano ridondanza all'interno dello schema, è stato deciso di mantenere i seguenti attributi:
\begin{itemize}
	\item \textbf{NumeroVisualizzati} nell'entità GENERE.
	\item \textbf{DurataComplessiva} e \textbf{NumeroStagioni} nell'entità SERIE.
	\item \textbf{NumeroEpisodi} nell'entità STAGIONE.
\end{itemize}

\section{Rimozione degli attributi composti}
CINEMA.Indirizzo composto da Via, CAP, Civico e Città viene scomposto usando disaggregazione in Ind{\_}Via, Ind{\_}CAP, Ind{\_}Civico e Ind{\_}Città.
\section{Traduzione di entità e associazioni in relazioni}
\subsection{Schema tradotto}
\begin{figure}[H]
	\centering
	\includegraphics[width=350pt]{ER/schemalogico/logicsx.png}
\end{figure}
\begin{figure}[H]
	\centering
	\includegraphics[width=350pt]{ER/schemalogico/logicdx.png}
\end{figure}

\subsection{Schema logico}
\begin{itemize} % Ho lasciato gli spazi per poter suddividere meglio a vista le varie "categorie" di Entità.
	\item \textbf{GENERI}(\underline{Nome}, Descrizione, NumeroVisualizzati)
	
	\item \textbf{CAST}(\underline{Codice}, Nome*)
	
	\item \textbf{FILM}(\underline{Codice}, Titolo, EtaLimite, Trama, Durata, CodiceCast:CAST)
	\item \textbf{CATEGORIZZAZIONI{\_}FILM}(\underline{NomeGenere}:GENERE, \underline{CodiceFilm}:FILM)
	
	\item \textbf{SERIE}(\underline{Codice}, Titolo, EtaLimite, Trama, DurataComplessiva, NumeroEpisodi)
	\item \textbf{STAGIONI}(\underline{CodiceSerie}:SERIE, \underline{NumeroStagione}, Sunto, CodiceCast:CAST)
	\item \textbf{EPISODI}(\underline{CodiceSerie}:STAGIONE, \underline{NumeroStagione}:STAGIONE, \underline{NumeroEpisodi}, DurataMin)
	\item \textbf{CATEGORIZZAZIONI{\_}SERIE}(\underline{NomeGenere}:GENERE, \underline{CodiceSerie}:SERIE)
	
	\item \textbf{SEZIONI}(\underline{Nome}, Dettaglio)
	
	\item \textbf{TEMPLATEPROMO}(\underline{CodicePromo}, PercentualeSconto)
	\item \textbf{PROMO}(\underline{CodiceTemplatePromo}:TEMPLATEPROMO, \underline{Scadenza})
	\item \textbf{SINGOLI}(\underline{CodiceTemplatePromo}:TEMPLATEPROMO, CodiceSerie:SERIE, CodiceFilm:FILM)
	\item \textbf{MULTIPLI}(\underline{CodiceTemplatePromo}:TEMPLATEPROMO)
	\item \textbf{PROMO{\_}GENERE}(\underline{NomeGenere}:GENERE, \underline{CodiceTemplateMultiplo}:MULTIPLO)
	
	\item \textbf{CINEMA}(\underline{Codice}, Nome, Ind{\_}Via, Ind{\_}CAP, Ind{\_}Civico, Ind{\_}Citta)
	
	\item \textbf{ACCOUNT}(\underline{Username}, Password, Nome, Cognome)
	\item \textbf{UTENTI}(\underline{Username}:ACCOUNT, TargaPremio*, DataNascita)
	\item \textbf{AMMINISTRATORI}(\underline{Username}:ACCOUNT, NumeroTelefono)
	\item \textbf{REGISTRATORI}(\underline{Username}:ACCOUNT, EmailCinema*, CodiceCinema:CINEMA)
	
	\item \textbf{MEMBRICAST}(\underline{Codice}, Nome, Cognome, DataNascita, TipoAttore, TipoRegista, DataDebuttoCarriera*, NomeArte*)
	\item \textbf{PARTECIPAZIONI{\_}CAST}(\underline{CodiceMembro}:MEMBROCAST, \underline{CodiceCast}:CAST)
	
	\item \textbf{PREFERENZE}(\underline{NomeGenere}:GENERE, \underline{UsernameUtente}:UTENTE)
	
	\item \textbf{TESSERE}(\underline{CodiceCinema}:CINEMA, \underline{UsernameUtente}:UTENTE, NumeroTessera, DataRinnovo)
	\item \textbf{PREMI{\_}TESSERA}(\underline{CodicePromo}:PROMO, \underline{Scadenza}:PROMO,\\  \underline{CodiceCinema}:TESSERA, \underline{UsernameUtente}:TESSERA)
	
	\item \textbf{RECSERIE}(\underline{UsernameUtente}:UTENTE, \underline{CodiceSerie}:SERIE, Titolo, Descrizione)
	\item \textbf{SEZIONAMENTI{\_}SERIE}(\underline{NomeSezione}:SEZIONE, \underline{UsernameUtente}:RECSERIE, \underline{CodiceRecSerie}:RECSERIE, Voto)
	\item \textbf{VALUTAZIONI{\_}SERIE}(\underline{UsernameUtenteValutato}:RECSERIE,\\  \underline{CodiceRecSerie}:RECSERIE, \underline{UsernameUtente}:UTENTE, Positiva)
	
	\item \textbf{RECFILM}(\underline{UsernameUtente}:UTENTE, \underline{CodiceFilm}:FILM, Titolo, Descrizione, VotoComplessivo)
	\item \textbf{SEZIONAMENTI{\_}FILM}(\underline{UsernameUtente}:RECFILM,\\ 
	\underline{CodiceRecFilm}:RECFILM, \underline{NomeSezione}:SEZIONE, Voto)
	\item \textbf{VALUTAZIONI{\_}FILM}(\underline{UsernameUtenteValutato}:RECFILM,\\ 
	\underline{CodiceRecFilm}:RECFILM, \underline{UsernameUtente}:UTENTE, Positiva)
	
	\item \textbf{VISUALIZZAZIONI{\_}EPISODIO}(\underline{UsernameUtente}:UTENTE,\\
	\underline{CodiceSerie}:EPISODIO,
	\underline{NumeroEpisodio}:EPISODIO, 
	\underline{NumeroStagione}:EPISODIO, Data)
	\item \textbf{VISUALIZZAZIONI{\_}FILM}(\underline{CodiceFilm}:FILM, \underline{UsernameUtente}:UTENTE)
	
	\item \textbf{RICHIESTE}(\underline{Numero}, Tipo, Titolo, AnnoUscita, Descrizione, Chiusa, UsernameUtente:UTENTE)
\end{itemize}

\section{Vincoli Intra-Relazionali}
\begin{itemize}
	\item GENERI:
	\begin{itemize}
		\item Vincoli di Dominio:
			\begin{itemize}
				\item Nome: VARCHAR(20) (non null)
				\item Descrizione: VARCHAR(500) (non null)
				\item NumeroVisualizzati: INT non negativo (non null)
			\end{itemize}
		\item Vincoli di Tupla: Nessuno
		\item Vincoli di Chiave: Nome
	\end{itemize}
	\item CAST:
	\begin{itemize}
		\item Vincoli di Dominio:
		\begin{itemize}
			\item Codice: INT (non null)
			\item Nome: VARCHAR(50)
		\end{itemize}
		\item Vincoli di Tupla: Nessuno
		\item Vincoli di Chiave: Codice
	\end{itemize}
	\item FILM:
	\begin{itemize}
		\item Vincoli di Dominio:
		\begin{itemize}
			\item Codice: INT (non null)
			\item Titolo: VARCHAR(50) (non null)
			\item EtaLimite: INT non negativo (non null)
			\item Trama: VARCHAR(500) (non null)
			\item Durata: INT positivo (non null)
			\item CodiceCast: INT (non null)
		\end{itemize}
		\item Vincoli di Tupla: Nessuno
		\item Vincoli di Chiave: Codice
	\end{itemize}
	\item CATEGORIZZAZIONE FILM:
	\begin{itemize}
		\item Vincoli di Dominio:
		\begin{itemize}
			\item NomeGenere: VARCHAR(20) (non null)
			\item CodiceFilm: INT (non null)
		\end{itemize}
		\item Vincoli di Tupla: Nessuno
		\item Vincoli di Chiave: NomeGenere, Codice
	\end{itemize}
	\item SERIE:
	\begin{itemize}
		\item Vincoli di Dominio:
		\begin{itemize}
			\item Codice: INT (non null)
			\item Titolo: VARCHAR(50) (non null)
			\item EtaLimite: INT (non null)
			\item Trama: VARCHAR(500) (non null)
			\item DurataComplessiva: INT (non null)
			\item NumeroEpisodi: INT (non null)
		\end{itemize}
		\item Vincoli di Tupla: Nessuno
		\item Vincoli di Chiave: Codice
	\end{itemize}
	\item STAGIONI:
	\begin{itemize}
		\item Vincoli di Dominio:
		\begin{itemize}
			\item CodiceSerie: INT (non null)
			\item NumeroStagioneL: INT (non null)
			\item Sunto: VARCHAR(500) (non null)
			\item CodiceCast: INT (non null)
		\end{itemize}
		\item Vincoli di Tupla: Nessuno
		\item Vincoli di Chiave: CodiceSerie, NumeroStagione
	\end{itemize}
	\item EPISODI:
	\begin{itemize}
		\item Vincoli di Dominio:
		\begin{itemize}
			\item CodiceSerie: INT (non null)
			\item NumeroStagione: INT (non null)
			\item NumeroEpisodio: INT (non null)
			\item DurataMin: INT (non null)
		\end{itemize}
		\item Vincoli di Tupla: Nessuno
		\item Vincoli di Chiave: CodiceSerie, NumeroStagione, NumeroEpisodio
	\end{itemize}
	\item CATEGORIZZAZIONE SERIE:
	\begin{itemize}
		\item Vincoli di Dominio:
		\begin{itemize}
			\item NomeGenere: VARCHAR(20) (non null)
			\item CodiceSerie: INT (non null)
		\end{itemize}
		\item Vincoli di Tupla: Nessuno
		\item Vincoli di Chiave: NomeGenere, CodiceSerie
	\end{itemize}
	\item SEZIONI:
	\begin{itemize}
		\item Vincoli di Dominio:
		\begin{itemize}
			\item Nome: CHAR(10) (non null)
			\item Dettaglio: VARCHAR(100) (non null)
		\end{itemize}
		\item Vincoli di Tupla: Nessuno
		\item Vincoli di Chiave: Nome
	\end{itemize}
	\item TEMPLATEPROMO:
	\begin{itemize}
		\item Vincoli di Dominio:
		\begin{itemize}
			\item CodicePromo: INT (non null)
			\item PercentualeSconto: INT compreso tra 0 e 100 (non null)
		\end{itemize}
		\item Vincoli di Tupla: Nessuno
		\item Vincoli di Chiave: CodicePromo
	\end{itemize}
	\item PROMO:
	\begin{itemize}
		\item Vincoli di Dominio:
		\begin{itemize}
			\item CodiceTemplatePromo: INT (non null)
			\item Scadenza: DATE (non null)
		\end{itemize}
		\item Vincoli di Tupla: Nessuno
		\item Vincoli di Chiave: CodiceTemplatePromo, Scadenza
	\end{itemize}
	\item SINGOLI:
	\begin{itemize}
		\item Vincoli di Dominio:
		\begin{itemize}
			\item CodiceTemplatePromo: INT (non null)
			\item CodiceSerie: INT (non null)
			\item CodiceFilm: INT (non null)
		\end{itemize}
		\item Vincoli di Tupla: Nessuno
		\item Vincoli di Chiave: CodiceTemplatePromo, CodiceSerie, CodiceFilm
	\end{itemize}
	\item MULTIPLI:
	\begin{itemize}
		\item Vincoli di Dominio:
		\begin{itemize}
			\item CodiceTemplatePromo: INT (non null)
		\end{itemize}
		\item Vincoli di Tupla: Nessuno
		\item Vincoli di Chiave: CodiceTemplatePromo
	\end{itemize}
	\item PROMO GENERE:
	\begin{itemize}
		\item Vincoli di Dominio:
		\begin{itemize}
			\item NomeGenere: VARCHAR(20) (non null)
			\item CodiceTemplateMultiplo: INT (non null)
		\end{itemize}
		\item Vincoli di Tupla: Nessuno
		\item Vincoli di Chiave: NomeGenere, CodiceTemplateMultiplo
	\end{itemize}
	\item CINEMA:
	\begin{itemize}
		\item Vincoli di Dominio:
		\begin{itemize}
			\item Codice: INT (non null)
			\item Nome: VARCHAR(50) (non null)
			\item IndVia: VARCHAR(30) (non null)
			\item IndCAP: CHAR(10) (non null)
			\item IndCivico: INT non negativo (non null)
			\item IndCitta: VARCHAR(30) (non null)
		\end{itemize}
		\item Vincoli di Tupla: Nessuno
		\item Vincoli di Chiave: Codice
	\end{itemize}
	\item ACCOUNT:
	\begin{itemize}
		\item Vincoli di Dominio:
		\begin{itemize}
			\item Username: CHAR(20) (non null)
			\item PASSWORD: CHAR(128) (non null)
			\item Nome: VARCHAR(20) (non null)
			\item Cognome: VARCHAR(20) (non null)
		\end{itemize}
		\item Vincoli di Tupla: Nessuno
		\item Vincoli di Chiave: Username
	\end{itemize}
	\item UTENTI:
	\begin{itemize}
		\item Vincoli di Dominio:
		\begin{itemize}
			\item Username: CHAR(20) (non null)
			\item TargaPremio: BOOLEAN (non null)
			\item DataNascita: DATE (non null)
		\end{itemize}
		\item Vincoli di Tupla: Nessuno
		\item Vincoli di Chiave: Username
	\end{itemize}
	\item AMMINISTRATORI:
	\begin{itemize}
		\item Vincoli di Dominio:
		\begin{itemize}
			\item Username: CHAR(20) (non null)
			\item NumeroTelefono: CHAR(10) (non null)
		\end{itemize}
		\item Vincoli di Tupla: Nessuno
		\item Vincoli di Chiave: Username
	\end{itemize}
	\item REGISTRATORI:
	\begin{itemize}
		\item Vincoli di Dominio:
		\begin{itemize}
			\item Username: CHAR(20) (non null)
			\item EmailCinema: VARCHAR(200)
			\item CodiceCinema: INT (non null)
		\end{itemize}
		\item Vincoli di Tupla: Nessuno
		\item Vincoli di Chiave: Username
	\end{itemize}
	\item MEMBRICAST:
	\begin{itemize}
		\item Vincoli di Dominio:
		\begin{itemize}
			\item Nome: VARCHAR(20) (non null)
			\item Cognome: VARCHAR(20) (non null)
			\item DataNascita: DATE (non null)
			\item TipoAttore: BOOLEAN (non null)
			\item TipoRegista: BOOLEAN (non null)
			\item DataDebuttoCarriera: DATE
			\item NomeArte: VARCHAR(30)	
		\end{itemize}
		\item Vincoli di Tupla: Nessuno
		\item Vincoli di Chiave: Codice
	\end{itemize}
	\item PARTECIPAZIONI CAST:
	\begin{itemize}
		\item Vincoli di Dominio:
		\begin{itemize}
			\item CodiceMembro: INT (non null)
			\item CodiceCast: INT (non null)
		\end{itemize}
		\item Vincoli di Tupla: Nessuno
		\item Vincoli di Chiave: CodiceMembro, CodiceCast
	\end{itemize}
	\item PREFERENZE:
	\begin{itemize}
		\item Vincoli di Dominio:
		\begin{itemize}
			\item NomeGenere: VARCHAR(20) (non null)
			\item UsernameUtente: CHAR(20) (non null)
		\end{itemize}
		\item Vincoli di Tupla: Nessuno
		\item Vincoli di Chiave: NomeGenere, UsernameUtente
	\end{itemize}
	\item TESSERE:
	\begin{itemize}
		\item Vincoli di Dominio:
		\begin{itemize}
			\item CodiceCinema: INT (non null)
			\item UsernameUtente: CHAR(20) (non null)
			\item NumeroTessera: INT (non null)
			\item DataRinnovo: DATE (non null)
		\end{itemize}
		\item Vincoli di Tupla: Nessuno
		\item Vincoli di Chiave: CodiceCinema, UsernameUtente
	\end{itemize}
	\item PREMI TESSERA:
	\begin{itemize}
		\item Vincoli di Dominio:
		\begin{itemize}
			\item CodicePromoPromo: INT (non null)
			\item Scadenza: DATE (non null)
			\item CodiceCinema: INT (non null)
			\item UsernameUtente: CHAR(20) (non null)
		\end{itemize}
		\item Vincoli di Tupla: Nessuno
		\item Vincoli di Chiave: CodicePromo, Scadenza, CodiceCinema, UsernameUtente
	\end{itemize}
	\item RECSERIE:
	\begin{itemize}
		\item Vincoli di Dominio:
		\begin{itemize}
			\item UsernameUtente: CHAR(20) (non null)
			\item CodiceSerie: INT (non null)
			\item Titolo: VARCHAR(50) (non null)
			\item Descrizione: VARCHAR(500) (non null)
			\item VotoComplessivo: INT compreso tra 0 e 5 (non null)
		\end{itemize}
		\item Vincoli di Tupla: Nessuno
		\item Vincoli di Chiave: UsernameUtente, CodiceSerie
	\end{itemize}
	\item SEZIONAMENTI SERIE:
	\begin{itemize}
		\item Vincoli di Dominio:
		\begin{itemize}
			\item NomeSezione: CHAR(10) (non null)
			\item UsernameUtente: CHAR(20) (non null)
			\item CodiceRecSerie: INT (non null)
			\item Voto: INT compreso tra 0 e 5 (non null)
		\end{itemize}
		\item Vincoli di Tupla: Nessuno
		\item Vincoli di Chiave: NomeSezione, UsernameUtente, CodiceRecSerie
	\end{itemize}
	\item VALUTAZIONI SERIE:
	\begin{itemize}
		\item Vincoli di Dominio:
		\begin{itemize}
			\item UsernameUtenteValutato: CHAR(20) (non null)
			\item CodiceRecSerie: INT (non null)
			\item UsernameUtente: CHAR(20) (non null)
			\item positiva: BOOLEAN (non null)
		\end{itemize}
		\item Vincoli di Tupla: Nessuno
		\item Vincoli di Chiave: UsernameUtenteValutato, CodiceRecSerie, UsernameUtente
	\end{itemize}
	\item RECFILM:
	\begin{itemize}
		\item Vincoli di Dominio:
		\begin{itemize}
			\item UsernameUtente: CHAR(20) (non null)
			\item CodiceFilm: INT (non null)
			\item Titolo: VARCHAR(50) (non null)
			\item Descrizione: VARCHAR(500) (non null)
			\item VotoComplessivo: INT compreso tra 0 e 5 (non null)
		\end{itemize}
		\item Vincoli di Tupla: Nessuno
		\item Vincoli di Chiave: UsernameUtente, CodiceFilm
	\end{itemize}
	\item SEZIONAMENTI FILM:
	\begin{itemize}
		\item Vincoli di Dominio:
		\begin{itemize}
			\item UsernameUtente: CHAR(20) (non null)
			\item CodiceRecFilm: INT (non null)
			\item NomeSezione: CHAR(10) (non null)
			\item Voto: INT compreso tra 0 e 5 (non null)
		\end{itemize}
		\item Vincoli di Tupla: Nessuno
		\item Vincoli di Chiave: UsernameUtente, CodiceRecFilm, NomeSezione
	\end{itemize}
	\item VALUTAZIONI FILM:
	\begin{itemize}
		\item Vincoli di Dominio:
		\begin{itemize}
			\item UsernameUtenteValutato: CHAR(20) (non null)
			\item CodiceRecFilm: INT (non null)
			\item UsernameUtente: CHAR(20) (non null)
			\item positiva: BOOLEAN (non null)
		\end{itemize}
		\item Vincoli di Tupla: Nessuno
		\item Vincoli di Chiave: UsernameUtenteValutato, CodiceRecFilm, UsernameUtente
	\end{itemize}
	\item VISUALIZZAZIONI EPISODIO:
	\begin{itemize}
		\item Vincoli di Dominio:
		\begin{itemize}
			\item UsernameUtente: CHAR(20) (non null)
			\item CodiceSerie: INT (non null)
			\item NumeroEpisodio: INT (non null)
			\item NumeroStagione: INT (non null)
			\item Data: DATE (non null)
		\end{itemize}
		\item Vincoli di Tupla: Nessuno
		\item Vincoli di Chiave: UsernameUtente, CodiceSerie, NumeroStagione, NumeroEpisodio
	\end{itemize}
	\item VISUALIZZAZIONI FILM:
	\begin{itemize}
		\item Vincoli di Dominio:
		\begin{itemize}
			\item CodiceFilm: INT (non null)
			\item UsernameUtente: CHAR(20) (non null)
		\end{itemize}
		\item Vincoli di Tupla: Nessuno
		\item Vincoli di Chiave: CodiceFilm, UsernameUtente
	\end{itemize}
	\item RICHIESTE:
	\begin{itemize}
		\item Vincoli di Dominio:
		\begin{itemize}
			\item Numero: INT (non null)
			\item Tipo: BOOLEAN (non null)
			\item Titolo: VARCHAR(100) (non null)
			\item AnnoUscita: DATE (non null)
			\item Descrizione: VARCHAR(100) (non null)
			\item Chiusa: BOOLEAN (non null)
			\item UsernameUtente: CHAR(20) (non null)
		\end{itemize}
		\item Vincoli di Tupla: Nessuno
		\item Vincoli di Chiave: Numero
	\end{itemize}
\end{itemize}

\section{Vincoli Inespressi}
\begin{itemize}
	\item Un ACCOUNT deve essere associato al suo corrispettivo tipo: AMMINISTRATORE, REGISTRATORE oppure UTENTE.
	\item Un UTENTE per recensire un FILM/SERIE deve aver visualizzato tale MULTIMEDIA.
	\item Un UTENTE, se minorenne, può visualizzare solo una parte di FILM/SERIE.
	\item Una TESSERA può ricevere una determinata PROMO, se essa é associata allo stesso CINEMA che registra tale TESSERA.
	\item Un UTENTE può effettuare una RICHIESTA per aggiungere un FILM/SERIE, se non disponibile nel Database.
\end{itemize}

\section{Traduzione delle operazioni in query SQL}
\begin{itemize}
	\item Op. 1 \\
	Questa operazione avviene alla registrazione di un utente, è un'operazione di inserimento, per evitare che ci siano account non associati a nessuno dei tre tipi di utenza previsti, avvengono due query in successione, una che registra l'account generico, e la seconda quello specifico.\\
	Prima query:
	\begin{verbatim}
		INSERT INTO account(Username, Password, Nome, Cognome) 
			VALUES(?, ?, ?, ?)
	\end{verbatim}
	Seconda query:
	\begin{verbatim}
		INSERT INTO utente(Username, TargaPremio, DataNascita)
			VALUES(?, ?, ?)
	\end{verbatim}
	\item Op. 2.1-3 \\ Questa operazione è divisa in due query diverse, una per controllare la validità della password senza reperire subito le informazioni dell'account, la seconda per reperirle una volta che l'utente è stato autenticato.\\
	Prima query:
	\begin{verbatim}
		SELECT Password FROM ACCOUNT WHERE Username = ?
	\end{verbatim}
	Seconda query:
	\begin{verbatim}
		SELECT
		account.Username AS ACC_NAME,
		registratore.username AS IS_REG,
		utente.Username AS IS_USR,
		amministratore.username AS IS_AMM 
		FROM 
		account 
		LEFT JOIN registratore 
			ON registratore.username = account.username 
		LEFT JOIN utente 
			ON utente.Username = account.Username 
		LEFT JOIN amministratore 
			ON amministratore.username = account.username
		WHERE account.username = ?
	\end{verbatim}
	Così strutturata, la query ritorna il valore del nome utente del tipo associato all'account, negli altri invece sarà presente il valore nullo, questo consente una gestione automatica della scelta del tipo di login da effettuare lato software.
	\item Op. 3.1-2: \\
	La scelta delle preferenze avviene in due passaggi: come prima cosa la totale rimozione di quelle precedentemente salvate, come secondo passaggio l'inserimento di quelle nuove.\\
	Prima query:
	\begin{verbatim}
		DELETE FROM preferenze WHERE UsernameUtente = ?
	\end{verbatim}
	Seconda query (avviene ciclicamente per quanti sono i generi totali):
	\begin{verbatim}
		INSERT INTO preferenze(NomeGenere, UsernameUtente)
			VALUES(?, ?)
	\end{verbatim}
	\item Op. 4.1-2:
	\begin{verbatim}
		SELECT * FROM film
		WHERE film.EtaLimite <= ?
	\end{verbatim}
	Dove ? viene sostituito con l'età dell'utente che effettua la richiesta, calcolata mediante la sua data di nascita nel software.
	\item Op. 4.3:
	\begin{verbatim}
		SELECT * FROM serie
		WHERE serie.EtaLimite <= ?
	\end{verbatim}
	Dove ? viene sostituito con l'età dell'utente che effettua la richiesta, calcolata mediante la sua data di nascita nel software.
	\item Op. 4.4:
	\begin{verbatim}
		SELECT 
			serie.Codice AS CodiceSerie,
			stagione.NumeroStagione,
			episodio.NumeroEpisodio,
			serie.Titolo AS TitoloSerie,
			serie.EtaLimite AS EtaLimiteSerie,
			serie.Trama AS TramaSerie,
			serie.DurataComplessiva AS DurataComplessivaSerie,
			serie.NumeroEpisodi AS NumeroEpisodiSerie,
			stagione.Sunto AS SuntoStagione,
			episodio.DurataMin AS DurataEpisodio,
			casting.Nome AS NomeCasting,
			membrocast.Codice AS CodiceMembroCast,
			membrocast.Nome AS NomeMembroCast,
			membrocast.Cognome AS CognomeMembroCast,
			membrocast.DataNascita AS DataNascitaMembroCast,
			membrocast.DataDebuttoCarriera 
				AS DataDebuttoCarrieraMembroCast,
			membrocast.NomeArte AS NomeArteMembroCast,
			membrocast.TipoAttore AS TipoAttoreMembroCast,
			membrocast.TipoRegista AS TipoRegistaMembroCast
		FROM serie
			JOIN stagione ON serie.Codice = stagione.CodiceSerie
			JOIN episodio ON episodio.NumeroStagione 
				= stagione.NumeroStagione
				AND episodio.CodiceSerie = stagione.CodiceSerie
			JOIN casting ON casting.Codice = stagione.CodiceCast
				AND episodio.CodiceSerie = stagione.CodiceSerie 
			JOIN partecipazione_cast 
				ON partecipazione_cast.CodiceCast = casting.Codice
			JOIN membrocast ON membrocast.Codice 
				= partecipazione_cast.CodiceMembro
		ORDER BY CodiceSerie
	\end{verbatim}
	\item Op. 5.1:\\
	Per la visualizzazione:
	\begin{verbatim}
		INSERT INTO visualizzazioni_film(CodiceFilm, UsernameUtente)
			VALUES(?, ?)
	\end{verbatim}
	Inoltre va gestito l'aggiornamento del numero complessivo di visualizzazioni per i generi associati:
	\begin{verbatim}
		UPDATE genere
		SET genere.NumeroVisualizzati = genere.NumeroVisualizzati + 1
		WHERE genere.Nome IN ( SELECT categorizzazione_film.NomeGenere
							   FROM categorizzazione_film
							   WHERE CodiceFilm = ? )
	\end{verbatim}
	Per la rimozione della visualizzazione (forget)
	\begin{verbatim}
		DELETE FROM visualizzazioni_film 
			WHERE CodiceFilm = ? AND UsernameUtente = ?
	\end{verbatim}
	L'aggiornamento delle visualizzazioni su genere per la rimozione della visualizzazione è analoga all'aggiunta ma con $-1$ invece che $+1$.
	\item Op. 5.2:\\
	Query per la visualizzazione:
	\begin{verbatim}
		INSERT INTO visualizzazioni_episodio(UsernameUtente,
											 CodiceSerie,
											 NumeroEpisodio,
											 NumeroStagione,
											 DATA)
			VALUES(?, ?, ?, ?, ?)
	\end{verbatim}
	Query per la rimozione della visualizzazione:
	\begin{verbatim}
		DELETE FROM visualizzazioni_episodio WHERE
			UsernameUtente = ? 
				AND CodiceSerie = ? 
				AND NumeroEpisodio = ? 
				AND NumeroStagione = ?
	\end{verbatim}
	L'aggiornamento delle visualizzazioni per l'aggiunta e la rimozione in genere è analoga a quella dei film, cambia solo la tabella della categorizzazione, quindi, vedi operazione precedente.
	\item Op. 6:\\
	Per recensire un film è necessario inserire tutte le sezioni della recensione e la recensione stessa, quindi:\\
	Query per l'inserimento della recensione complessiva.
	\begin{verbatim}
		INSERT INTO recfilm(CodiceFilm, UsernameUtente,
		 Titolo, Descrizione)
			VALUES (?,?,?,?)
	\end{verbatim}
	Query per le sezioni (avviene per ogni sezione recensita) dopo l'inserimento della recensione stessa:
	\begin{verbatim}
		INSERT INTO sezionamento_film (NomeSezione, UsernameUtente,
		 CodiceRecFilm, Voto)
			VALUES (?,?,?,?)
	\end{verbatim}
	\item Op. 7.1:
	\begin{verbatim}
		SELECT
			recensioni_totali.UsernameUtente,
			recensioni_totali.CodiceFilm,
			film.Titolo AS TitoloFilm,
			recensioni_totali.CodiceSerie,
			serie.Titolo AS TitoloSerie,
			recensioni_totali.TitoloRecensione,
			recensioni_totali.DescrizioneRecensione,
			recensioni_totali.VotoComplessivoRecensione
		FROM (
			SELECT
				UsernameUtente,
				CodiceSerie,
				NULL AS CodiceFilm,
				Titolo AS TitoloSerie,
				NULL AS TitoloFilm,
				Titolo AS TitoloRecensione,
				Descrizione AS DescrizioneRecensione,
				VotoComplessivo AS VotoComplessivoRecensione
			FROM
				recserie
			UNION ALL
			SELECT
				UsernameUtente,
				NULL AS CodiceSerie,
				CodiceFilm,
				NULL AS TitoloSerie,
				Titolo AS TitoloFilm,
				Titolo AS TitoloRecensione,
				Descrizione AS DescrizioneRecensione,
				VotoComplessivo AS VotoComplessivoRecensione
			FROM
				recfilm
		) AS recensioni_totali
		LEFT JOIN film ON recensioni_totali.CodiceFilm = film.Codice
		LEFT JOIN serie ON recensioni_totali.CodiceSerie = serie.Codice
		WHERE UsernameUtente = ?
	\end{verbatim}
	\item Op. 7.2-3: \label{op:7.2-3}\\
	Operazione analoga per le serie ma con CodiceRecSerie invece che CodiceRecFilm.
	\begin{verbatim}
		INSERT INTO valutazione_film (UsernameUtenteValutato, CodiceRecFilm,
									  UsernameUtente, positiva)
			VALUES (?,?,?,?)
	\end{verbatim}
	\item Op. 7.4:
	\begin{verbatim}
		SELECT recserie.*, serie.Titolo AS TitoloSerie
		FROM recserie JOIN serie ON recserie.CodiceSerie = serie.Codice
		WHERE recserie.CodiceSerie = ?
	\end{verbatim}
	\item Op. 7.5: Semplice aggiunta di:
	\begin{verbatim}
		ON DUPLICATE KEY UPDATE Positiva = VALUES(Positiva)
	\end{verbatim}
	Alle query 7.2-3 (\ref{op:7.2-3})
	\item Op. 8.1:\\
	Utilizziamo il campo introdotto a seguito dello studio di ridondanza.
	\begin{verbatim}
		SELECT
			genere.Nome,
			genere.Descrizione,
			genere.NumeroVisualizzati
		FROM genere
		ORDER BY genere.NumeroVisualizzati DESC
	\end{verbatim}
	\item Op. 9.1:
	\begin{verbatim}
		SELECT UsernameUtenteValutato,
			SUM(
				CASE WHEN Positiva = TRUE THEN 1 ELSE -1 END) 
				/ 
				COUNT(*) AS MediaValutazione
		FROM (SELECT UsernameUtenteValutato, Positiva FROM valutazione_film
			UNION ALL
			SELECT UsernameUtenteValutato, Positiva FROM valutazione_serie) 
		AS AllReviews
		GROUP BY UsernameUtenteValutato
		ORDER BY MediaValutazione ASC
		LIMIT 5
	\end{verbatim}
	\item Op. 9.2:
	\begin{verbatim}
		SELECT UsernameUtente, SUM(NumeroValutazioni) AS TotaleValutazioni
			FROM (
		 		SELECT UsernameUtente, COUNT(*) AS NumeroValutazioni
		 		FROM recfilm
		 		GROUP BY UsernameUtente
		 		UNION ALL
		 		SELECT UsernameUtente, COUNT(*) AS NumeroValutazioni
		 		FROM recserie
		 		GROUP BY UsernameUtente
		 	) AS RecensioniTotali
		 GROUP BY UsernameUtente
		 ORDER BY TotaleValutazioni DESC
		 LIMIT 5
	\end{verbatim}
	\item Op. 10:
	\begin{verbatim}
		INSERT INTO premi_tessera(CodicePromoPromo,
			Scadenza, CodiceCinema, UsernameUtente)
		VALUES (?, ?, ?, ?)
		ON DUPLICATE KEY UPDATE CodicePromoPromo = CodicePromoPromo
	\end{verbatim}
	\item Op. 11.1:
	\begin{verbatim}
		INSERT INTO film (Titolo, EtaLimite, Trama, Durata, CodiceCast)
			VALUES (?, ?, ?, ?, ?)
	\end{verbatim}
	\item Op. 11.2:
	\begin{verbatim}
		INSERT INTO membrocast (Nome, 
			Cognome, 
			DataNascita, 
			TipoAttore, 
			TipoRegista,
			DataDebuttoCarriera, 
			NomeArte)
		VALUES (?, ?, ?, ?, ?, ?, ?)
	\end{verbatim}
	\item Op. 11.3:
	\begin{verbatim}
		INSERT INTO
		stagione (CodiceSerie, NumeroStagione, Sunto, CodiceCast)
		VALUES (?, ?, ?, ?)
	\end{verbatim}
	\item Op. 11.4:
	\begin{verbatim}
		INSERT
		INTO episodio (NumeroEpisodio,
		CodiceSerie,
		NumeroStagione,
		DurataMin)
		VALUES (?, ?, ?, ?)
	\end{verbatim}
	\item Op. 12:\\
	Operazione eseguita dal registratore.
	\begin{verbatim}
		INSERT INTO tessera(CodiceCinema, UsernameUtente,
		 NumeroTessera, DataRinnovo)
			VALUES (?,?,?,?)
	\end{verbatim}
\end{itemize}

\chapter{Progettazione dell'applicazione}
\section{Architettura generale}
\subsection{Infrastruttura}
L'applicazione e la relativa interfaccia è stata realizzata con il linguaggio \textbf{Java 17} con l'intenzione di renderla multi-piattaforma. Come DBMS invece, abbiamo scelto l'alternativa open-source a MySQL ma completamente compatibile, ovvero \textbf{MariaDB}, la cui installazione può essere effettuata sia in locale che su un server remoto per potervi accedere. Per il collegamento a questo DBMS è stato usata la libreria \textbf{JDBC} con il \href{https://mvnrepository.com/artifact/org.mariadb.jdbc/mariadb-java-client}{driver apposito per MariaDB} di tipo 4.\\
La compilazione e la gestione delle dipendenze è stata gestita con \textbf{Gradle} e il versioning con \textbf{Git}, le repository remote sono salvaguardate su \textbf{Github}.
\subsection{Pattern architetturale}
Tutta l'infrastruttura del software si basa sul modello \textbf{MVC} (Model-View-Controller) che rende indipendente la controparte grafica del software (realizzata con la libreria del jdk \textbf{Swing}) a quella di modellazione (che consiste nella definizione delle classi che rispecchiano la struttura del database progettato nelle sezioni precedenti).

\section{Interfaccia}
\subsection{Login}
Come specificato in precedenza, per l'interfaccia è stata utilizzata la libreria Java Swing, all'aprirsi dell'applicativo, la prima schermata ad apparire è quella di Login, che consente di inserire Username e Password per effettuare l'accesso, il tipo di utente che effettua il login è stabilito tramite query, quindi avviene in automatico senza doverlo specificare esplicitamente. Per come è stato progettato il database, non è possibile avere utenti con lo stesso username, anche se di tipo diverso.\\
Questa è la procedura generica che è possibile effettuare qualora la connessione al database locale o remoto avvenga con successo, altrimenti, verrà visualizzato un messaggio di avviso, tutti i componenti dell'interfaccia verranno disabilitati e l'applicazione verrà chiusa.
\begin{figure}[H]
	\centering
	\includegraphics[width=300pt]{appimg/loginjw.png}
	\caption{Pagina di login.}
\end{figure}
\subsection{Registratore}
Se viene effettuato l'accesso con un account di amministrazione, si verrà portati nella sezione che consente di visualizzare le proprie informazioni di profilo (centro pagina), nella porzione superiore dell'interfaccia invece, vi è una barra di navigazione che consente di spostarsi nell'area che permette di registrare utenti tesserati al proprio cinema, le cui informazioni sono anch'esse contenute in quest'ultima.
\begin{figure}[H]
	\centering
	\includegraphics[width=300pt]{appimg/reginfojw.png}
	\caption{Pagina iniziale del registratore.}
\end{figure}
Nella seconda pagina accessibile sono presentati in ordine le informazioni del cinema a cui il registratore che ha effettuato l'accesso è associato, e un pannello di registrazione di una tessera per il suddetto cinema.\\
Della tessera è necessario inserire il nome utente dell'intestatario, la data di rinnovo prevista attraverso il selezionatore, ed un numero di tessera che deve essere univoco nel cinema di appartenenza (altrimenti la registrazione non potrà avvenire e verrà richiesta chiaramente una modifica dei campi).
\begin{figure}[H]
	\centering
	\includegraphics[width=300pt]{appimg/regregjw.png}
	\caption{Pagina di registrazione del registratore.}
\end{figure}
Quando una tessera viene aggiunta correttamente al DB verrà mostrato un prompt verde a schermo che ne conferma il successo, i campi verranno azzerati e sarà possibile procedere con un'altra registrazione.
\subsection{Utente}
Una volta effettuato l'accesso con un account utente, l'utente viene portato nella homepage che consente di visualizzare i film e le serie TV disponibili, filtrati in base all'età dell'utente o a un'età scelta dall'utente stesso. 
\begin{figure}[H]
	\centering
	\includegraphics[width=375pt]{appimg/userhomepage.png}
	\caption{Pagina di apertura(homepage) dell'utente.}
\end{figure}

In questa sezione, l'utente può esplorare un elenco di film e serie TV e, cliccando su un titolo, accede a una pagina dettagliata che mostra le informazioni principali come il cast, la durata e le recensioni lasciate da altri utenti.
\begin{figure}[H]
	\centering
	\includegraphics[width=300pt]{appimg/userfilmdetails.png}
	\caption{Pagina di dettaglio di un film.}
\end{figure}

Quando l'utente visualizza la pagina dettagliata di un film o di una serie TV, ha la possibilità di cliccare su una recensione di un altro utente per vedere in dettaglio il voto assegnato nelle diverse sezioni della recensione. 
\begin{figure}[H]
	\centering
	\includegraphics[width=300pt]{appimg/userrecdetails.png}
	\caption{Pagina di dettaglio della recensione.}
\end{figure}

Se l'utente ha visualizzato tutti gli episodi di una serie o ha visto un film, viene abilitato il bottone per recensire il contenuto. Cliccando su questo bottone, l'utente accede a una pagina dove può inserire il titolo della recensione, una breve descrizione e valutare diverse sezioni del film o della serie TV tramite voti numerici.
\begin{figure}[H]
	\centering
	\includegraphics[width=300pt]{appimg/userwriterec.png}
	\caption{Pagina di scrittura della recensione.}
\end{figure}

L'utente ha anche accesso a una sezione dove può visualizzare tutte le proprie recensioni effettuate in passato. Ognuna delle quali può essere cliccata per visualizzarne i dettagli. 

Inoltre, l'utente può accedere a una pagina che mostra le proprie informazioni personali, come il nome utente, il nome, il cognome e altri dettagli, in una modalità simile alla visualizzazione del profilo dell'amministratore, ma con le informazioni personali dell'utente.
\begin{figure}[H]
	\centering
	\includegraphics[width=300pt]{appimg/userprofilepage.png}
	\caption{Pagina di dettaglio della recensione.}
\end{figure}
\subsection{Amministratore}
Una volta effettuato l'accesso con un account amministratore, l'amministratore viene portato nella dashboard principale, che consente di gestire i film e le serie TV disponibili, il cast, le classifiche, le tessere, le promozioni, le richieste degli utenti, i registratori, gli utenti e le proprie informazioni personali. \\
\begin{figure}[H]
	\centering
	\includegraphics[width=375pt]{appimg/adminImages/homepage.png}
	\caption{Homepage dell'amministratore}
\end{figure}

In questa sezione, l'amministratore può visualizzare un elenco di film. Cliccando su un titolo, accede a una pagina dettagliata che mostra le informazioni principali come il cast, la durata ed altri dettagli tecnici. L'amministratore può modificare o aggiornare queste informazioni ed aggiungere nuovi titoli al catalogo. \\
\begin{figure}[H]
	\centering
	\includegraphics[width=300pt]{appimg/adminImages/dettaglifilm.png}
	\caption{Pagina di dettaglio di un film}
\end{figure}

Nella sezione delle serie TV, l'amministratore può visualizzare un elenco di serie. Cliccando su un titolo, accede a una pagina dettagliata che mostra le informazioni principali come le stagioni, gli episodi di ogni stagione con la loro durata ed altri dettagli tecnici. L'amministratore può modificare o aggiornare queste informazioni ed aggiungere nuove serie,stagioni ed episodi al catalogo. \\
\begin{figure}[H]
	\centering
	\includegraphics[width=375pt]{appimg/adminImages/sezioneserie.png}
	\caption{Pagina delle Serie TV}
\end{figure}
\begin{figure}[H]
	\centering
	\includegraphics[width=300pt]{appimg/adminImages/dettagliserie.png}
	\caption{Pagina di dettaglio di una Serie TV}
\end{figure}

L'amministratore ha accesso a una sezione dedicata al cast, dove può visualizzare e gestire gli attori e le attrici associati ai vari film e serie TV. Cliccando su un cast, l'amministratore accede a una pagina dettagliata che mostra le informazioni principali dell'attore o attrice che partecipano. Anche queste informazioni possono essere modificate o aggiornate. \\
\begin{figure}[H]
	\centering
	\includegraphics[width=375pt]{appimg/adminImages/sezionecast.png}
	\caption{Pagina dei Cast, Attori e Registi}
\end{figure}

Nella sezione delle classifiche, l'amministratore può visualizzare e gestire le graduatorie degli utenti in base a diversi criteri, come il numero di recensioni o la loro qualità. Inoltre, l'amministratore può assegnare in blocco una promozione per i primi 5 utenti con il maggior numero di recensioni. \\
\begin{figure}[H]
	\centering
	\includegraphics[width=375pt]{appimg/adminImages/sezioneclassifiche.png}
	\caption{Pagina delle Classifiche}
\end{figure}

L'amministratore ha anche la possibilità di gestire le tessere degli utenti, che includono informazioni come il numero della tessera, il nome dell'utente e la sua data di rinnovo. L'amministratore ha la possibilità assegnare delle promo già esistenti alle tessere, a seconda delle necessità. \\
\begin{figure}[H]
	\centering
	\includegraphics[width=375pt]{appimg/adminImages/sezionetessere.png}
	\caption{Pagina delle Tessere}
\end{figure}

La sezione delle promozioni consente all'amministratore di creare e gestire le promozioni e gli sconti disponibili per gli utenti. Può aggiungere Promo Template, Singole, Multiple, su Genere e Promo con Scadenza. Inoltre, può eliminare Template e Promo con Scadenza già esistenti. \\
\begin{figure}[H]
	\centering
	\includegraphics[width=375pt]{appimg/adminImages/sezionepromo.png}
	\caption{Pagina delle Promo}
\end{figure}

Nella sezione delle richieste, l'amministratore può visualizzare e gestire le richieste degli utenti, che possono riguardare nuovi titoli da aggiungere al catalogo. Per ogni richiesta, l'amministratore può visulizzare il tipo di richiesta, il titolo, l'anno di uscita e tutte le informazioni necessarie per aggiungere la multimedia al catalogo dei film o serie TV. \\
\begin{figure}[H]
	\centering
	\includegraphics[width=375pt]{appimg/adminImages/sezionerichieste.png}
	\caption{Pagina delle Richieste}
\end{figure}

Nella sezione dei registratori, l'amministratore ha la possibilità di visualizzare i registratori ed i cinema disponibili con le loro informazioni, nonché di aggiungerli o eliminarli secondo le necessità. \\
\begin{figure}[H]
	\centering
	\includegraphics[width=375pt]{appimg/adminImages/sezioneregistratori.png}
	\caption{Pagina dei Registratori}
\end{figure}

Nella sezione utenti, l'amministratore può visualizzare le informazioni degli utenti registrati sulla piattaforma e, se necessario, eliminare quelli ritenuti inefficaci nella scrittura di recensioni, basandosi sulla classifica disponibile nella sezione dedicata. \\
\begin{figure}[H]
	\centering
	\includegraphics[width=375pt]{appimg/adminImages/sezioneutenti.png}
	\caption{Pagina degli Utenti}
\end{figure}

L'amministratore ha anche accesso a una pagina che mostra le proprie informazioni personali, come il nome, il cognome, il nome utente e il numero di telefono. Inoltre, da questa sezione, può aggiungere ed eliminare un Genere e una Sezione di Valutazione. \\
\begin{figure}[H]
	\centering
	\includegraphics[width=375pt]{appimg/adminImages/sezioneprofilo.png}
	\caption{Pagina delle Richieste}
\end{figure}

\end{document}